\documentclass[11pt]{article}

    \usepackage[breakable]{tcolorbox}
    \usepackage{parskip} % Stop auto-indenting (to mimic markdown behaviour)
    

    % Basic figure setup, for now with no caption control since it's done
    % automatically by Pandoc (which extracts ![](path) syntax from Markdown).
    \usepackage{graphicx}
    % Maintain compatibility with old templates. Remove in nbconvert 6.0
    \let\Oldincludegraphics\includegraphics
    % Ensure that by default, figures have no caption (until we provide a
    % proper Figure object with a Caption API and a way to capture that
    % in the conversion process - todo).
    \usepackage{caption}
    \DeclareCaptionFormat{nocaption}{}
    \captionsetup{format=nocaption,aboveskip=0pt,belowskip=0pt}

    \usepackage{float}
    \floatplacement{figure}{H} % forces figures to be placed at the correct location
    \usepackage{xcolor} % Allow colors to be defined
    \usepackage{enumerate} % Needed for markdown enumerations to work
    \usepackage{geometry} % Used to adjust the document margins
    \usepackage{amsmath} % Equations
    \usepackage{amssymb} % Equations
    \usepackage{textcomp} % defines textquotesingle
    % Hack from http://tex.stackexchange.com/a/47451/13684:
    \AtBeginDocument{%
        \def\PYZsq{\textquotesingle}% Upright quotes in Pygmentized code
    }
    \usepackage{upquote} % Upright quotes for verbatim code
    \usepackage{eurosym} % defines \euro

    \usepackage{iftex}
    \ifPDFTeX
        \usepackage[T1]{fontenc}
        \IfFileExists{alphabeta.sty}{
              \usepackage{alphabeta}
          }{
              \usepackage[mathletters]{ucs}
              \usepackage[utf8x]{inputenc}
          }
    \else
        \usepackage{fontspec}
        \usepackage{unicode-math}
    \fi

    \usepackage{fancyvrb} % verbatim replacement that allows latex
    \usepackage{grffile} % extends the file name processing of package graphics
                         % to support a larger range
    \makeatletter % fix for old versions of grffile with XeLaTeX
    \@ifpackagelater{grffile}{2019/11/01}
    {
      % Do nothing on new versions
    }
    {
      \def\Gread@@xetex#1{%
        \IfFileExists{"\Gin@base".bb}%
        {\Gread@eps{\Gin@base.bb}}%
        {\Gread@@xetex@aux#1}%
      }
    }
    \makeatother
    \usepackage[Export]{adjustbox} % Used to constrain images to a maximum size
    \adjustboxset{max size={0.9\linewidth}{0.9\paperheight}}

    % The hyperref package gives us a pdf with properly built
    % internal navigation ('pdf bookmarks' for the table of contents,
    % internal cross-reference links, web links for URLs, etc.)
    \usepackage{hyperref}
    % The default LaTeX title has an obnoxious amount of whitespace. By default,
    % titling removes some of it. It also provides customization options.
    \usepackage{titling}
    \usepackage{longtable} % longtable support required by pandoc >1.10
    \usepackage{booktabs}  % table support for pandoc > 1.12.2
    \usepackage{array}     % table support for pandoc >= 2.11.3
    \usepackage{calc}      % table minipage width calculation for pandoc >= 2.11.1
    \usepackage[inline]{enumitem} % IRkernel/repr support (it uses the enumerate* environment)
    \usepackage[normalem]{ulem} % ulem is needed to support strikethroughs (\sout)
                                % normalem makes italics be italics, not underlines
    \usepackage{mathrsfs}
    

    
    % Colors for the hyperref package
    \definecolor{urlcolor}{rgb}{0,.145,.698}
    \definecolor{linkcolor}{rgb}{.71,0.21,0.01}
    \definecolor{citecolor}{rgb}{.12,.54,.11}

    % ANSI colors
    \definecolor{ansi-black}{HTML}{3E424D}
    \definecolor{ansi-black-intense}{HTML}{282C36}
    \definecolor{ansi-red}{HTML}{E75C58}
    \definecolor{ansi-red-intense}{HTML}{B22B31}
    \definecolor{ansi-green}{HTML}{00A250}
    \definecolor{ansi-green-intense}{HTML}{007427}
    \definecolor{ansi-yellow}{HTML}{DDB62B}
    \definecolor{ansi-yellow-intense}{HTML}{B27D12}
    \definecolor{ansi-blue}{HTML}{208FFB}
    \definecolor{ansi-blue-intense}{HTML}{0065CA}
    \definecolor{ansi-magenta}{HTML}{D160C4}
    \definecolor{ansi-magenta-intense}{HTML}{A03196}
    \definecolor{ansi-cyan}{HTML}{60C6C8}
    \definecolor{ansi-cyan-intense}{HTML}{258F8F}
    \definecolor{ansi-white}{HTML}{C5C1B4}
    \definecolor{ansi-white-intense}{HTML}{A1A6B2}
    \definecolor{ansi-default-inverse-fg}{HTML}{FFFFFF}
    \definecolor{ansi-default-inverse-bg}{HTML}{000000}

    % common color for the border for error outputs.
    \definecolor{outerrorbackground}{HTML}{FFDFDF}

    % commands and environments needed by pandoc snippets
    % extracted from the output of `pandoc -s`
    \providecommand{\tightlist}{%
      \setlength{\itemsep}{0pt}\setlength{\parskip}{0pt}}
    \DefineVerbatimEnvironment{Highlighting}{Verbatim}{commandchars=\\\{\}}
    % Add ',fontsize=\small' for more characters per line
    \newenvironment{Shaded}{}{}
    \newcommand{\KeywordTok}[1]{\textcolor[rgb]{0.00,0.44,0.13}{\textbf{{#1}}}}
    \newcommand{\DataTypeTok}[1]{\textcolor[rgb]{0.56,0.13,0.00}{{#1}}}
    \newcommand{\DecValTok}[1]{\textcolor[rgb]{0.25,0.63,0.44}{{#1}}}
    \newcommand{\BaseNTok}[1]{\textcolor[rgb]{0.25,0.63,0.44}{{#1}}}
    \newcommand{\FloatTok}[1]{\textcolor[rgb]{0.25,0.63,0.44}{{#1}}}
    \newcommand{\CharTok}[1]{\textcolor[rgb]{0.25,0.44,0.63}{{#1}}}
    \newcommand{\StringTok}[1]{\textcolor[rgb]{0.25,0.44,0.63}{{#1}}}
    \newcommand{\CommentTok}[1]{\textcolor[rgb]{0.38,0.63,0.69}{\textit{{#1}}}}
    \newcommand{\OtherTok}[1]{\textcolor[rgb]{0.00,0.44,0.13}{{#1}}}
    \newcommand{\AlertTok}[1]{\textcolor[rgb]{1.00,0.00,0.00}{\textbf{{#1}}}}
    \newcommand{\FunctionTok}[1]{\textcolor[rgb]{0.02,0.16,0.49}{{#1}}}
    \newcommand{\RegionMarkerTok}[1]{{#1}}
    \newcommand{\ErrorTok}[1]{\textcolor[rgb]{1.00,0.00,0.00}{\textbf{{#1}}}}
    \newcommand{\NormalTok}[1]{{#1}}

    % Additional commands for more recent versions of Pandoc
    \newcommand{\ConstantTok}[1]{\textcolor[rgb]{0.53,0.00,0.00}{{#1}}}
    \newcommand{\SpecialCharTok}[1]{\textcolor[rgb]{0.25,0.44,0.63}{{#1}}}
    \newcommand{\VerbatimStringTok}[1]{\textcolor[rgb]{0.25,0.44,0.63}{{#1}}}
    \newcommand{\SpecialStringTok}[1]{\textcolor[rgb]{0.73,0.40,0.53}{{#1}}}
    \newcommand{\ImportTok}[1]{{#1}}
    \newcommand{\DocumentationTok}[1]{\textcolor[rgb]{0.73,0.13,0.13}{\textit{{#1}}}}
    \newcommand{\AnnotationTok}[1]{\textcolor[rgb]{0.38,0.63,0.69}{\textbf{\textit{{#1}}}}}
    \newcommand{\CommentVarTok}[1]{\textcolor[rgb]{0.38,0.63,0.69}{\textbf{\textit{{#1}}}}}
    \newcommand{\VariableTok}[1]{\textcolor[rgb]{0.10,0.09,0.49}{{#1}}}
    \newcommand{\ControlFlowTok}[1]{\textcolor[rgb]{0.00,0.44,0.13}{\textbf{{#1}}}}
    \newcommand{\OperatorTok}[1]{\textcolor[rgb]{0.40,0.40,0.40}{{#1}}}
    \newcommand{\BuiltInTok}[1]{{#1}}
    \newcommand{\ExtensionTok}[1]{{#1}}
    \newcommand{\PreprocessorTok}[1]{\textcolor[rgb]{0.74,0.48,0.00}{{#1}}}
    \newcommand{\AttributeTok}[1]{\textcolor[rgb]{0.49,0.56,0.16}{{#1}}}
    \newcommand{\InformationTok}[1]{\textcolor[rgb]{0.38,0.63,0.69}{\textbf{\textit{{#1}}}}}
    \newcommand{\WarningTok}[1]{\textcolor[rgb]{0.38,0.63,0.69}{\textbf{\textit{{#1}}}}}


    % Define a nice break command that doesn't care if a line doesn't already
    % exist.
    \def\br{\hspace*{\fill} \\* }
    % Math Jax compatibility definitions
    \def\gt{>}
    \def\lt{<}
    \let\Oldtex\TeX
    \let\Oldlatex\LaTeX
    \renewcommand{\TeX}{\textrm{\Oldtex}}
    \renewcommand{\LaTeX}{\textrm{\Oldlatex}}
    % Document parameters
    % Document title
    \title{deeplearning\_segmentation}
    
    
    
    
    
% Pygments definitions
\makeatletter
\def\PY@reset{\let\PY@it=\relax \let\PY@bf=\relax%
    \let\PY@ul=\relax \let\PY@tc=\relax%
    \let\PY@bc=\relax \let\PY@ff=\relax}
\def\PY@tok#1{\csname PY@tok@#1\endcsname}
\def\PY@toks#1+{\ifx\relax#1\empty\else%
    \PY@tok{#1}\expandafter\PY@toks\fi}
\def\PY@do#1{\PY@bc{\PY@tc{\PY@ul{%
    \PY@it{\PY@bf{\PY@ff{#1}}}}}}}
\def\PY#1#2{\PY@reset\PY@toks#1+\relax+\PY@do{#2}}

\@namedef{PY@tok@w}{\def\PY@tc##1{\textcolor[rgb]{0.73,0.73,0.73}{##1}}}
\@namedef{PY@tok@c}{\let\PY@it=\textit\def\PY@tc##1{\textcolor[rgb]{0.24,0.48,0.48}{##1}}}
\@namedef{PY@tok@cp}{\def\PY@tc##1{\textcolor[rgb]{0.61,0.40,0.00}{##1}}}
\@namedef{PY@tok@k}{\let\PY@bf=\textbf\def\PY@tc##1{\textcolor[rgb]{0.00,0.50,0.00}{##1}}}
\@namedef{PY@tok@kp}{\def\PY@tc##1{\textcolor[rgb]{0.00,0.50,0.00}{##1}}}
\@namedef{PY@tok@kt}{\def\PY@tc##1{\textcolor[rgb]{0.69,0.00,0.25}{##1}}}
\@namedef{PY@tok@o}{\def\PY@tc##1{\textcolor[rgb]{0.40,0.40,0.40}{##1}}}
\@namedef{PY@tok@ow}{\let\PY@bf=\textbf\def\PY@tc##1{\textcolor[rgb]{0.67,0.13,1.00}{##1}}}
\@namedef{PY@tok@nb}{\def\PY@tc##1{\textcolor[rgb]{0.00,0.50,0.00}{##1}}}
\@namedef{PY@tok@nf}{\def\PY@tc##1{\textcolor[rgb]{0.00,0.00,1.00}{##1}}}
\@namedef{PY@tok@nc}{\let\PY@bf=\textbf\def\PY@tc##1{\textcolor[rgb]{0.00,0.00,1.00}{##1}}}
\@namedef{PY@tok@nn}{\let\PY@bf=\textbf\def\PY@tc##1{\textcolor[rgb]{0.00,0.00,1.00}{##1}}}
\@namedef{PY@tok@ne}{\let\PY@bf=\textbf\def\PY@tc##1{\textcolor[rgb]{0.80,0.25,0.22}{##1}}}
\@namedef{PY@tok@nv}{\def\PY@tc##1{\textcolor[rgb]{0.10,0.09,0.49}{##1}}}
\@namedef{PY@tok@no}{\def\PY@tc##1{\textcolor[rgb]{0.53,0.00,0.00}{##1}}}
\@namedef{PY@tok@nl}{\def\PY@tc##1{\textcolor[rgb]{0.46,0.46,0.00}{##1}}}
\@namedef{PY@tok@ni}{\let\PY@bf=\textbf\def\PY@tc##1{\textcolor[rgb]{0.44,0.44,0.44}{##1}}}
\@namedef{PY@tok@na}{\def\PY@tc##1{\textcolor[rgb]{0.41,0.47,0.13}{##1}}}
\@namedef{PY@tok@nt}{\let\PY@bf=\textbf\def\PY@tc##1{\textcolor[rgb]{0.00,0.50,0.00}{##1}}}
\@namedef{PY@tok@nd}{\def\PY@tc##1{\textcolor[rgb]{0.67,0.13,1.00}{##1}}}
\@namedef{PY@tok@s}{\def\PY@tc##1{\textcolor[rgb]{0.73,0.13,0.13}{##1}}}
\@namedef{PY@tok@sd}{\let\PY@it=\textit\def\PY@tc##1{\textcolor[rgb]{0.73,0.13,0.13}{##1}}}
\@namedef{PY@tok@si}{\let\PY@bf=\textbf\def\PY@tc##1{\textcolor[rgb]{0.64,0.35,0.47}{##1}}}
\@namedef{PY@tok@se}{\let\PY@bf=\textbf\def\PY@tc##1{\textcolor[rgb]{0.67,0.36,0.12}{##1}}}
\@namedef{PY@tok@sr}{\def\PY@tc##1{\textcolor[rgb]{0.64,0.35,0.47}{##1}}}
\@namedef{PY@tok@ss}{\def\PY@tc##1{\textcolor[rgb]{0.10,0.09,0.49}{##1}}}
\@namedef{PY@tok@sx}{\def\PY@tc##1{\textcolor[rgb]{0.00,0.50,0.00}{##1}}}
\@namedef{PY@tok@m}{\def\PY@tc##1{\textcolor[rgb]{0.40,0.40,0.40}{##1}}}
\@namedef{PY@tok@gh}{\let\PY@bf=\textbf\def\PY@tc##1{\textcolor[rgb]{0.00,0.00,0.50}{##1}}}
\@namedef{PY@tok@gu}{\let\PY@bf=\textbf\def\PY@tc##1{\textcolor[rgb]{0.50,0.00,0.50}{##1}}}
\@namedef{PY@tok@gd}{\def\PY@tc##1{\textcolor[rgb]{0.63,0.00,0.00}{##1}}}
\@namedef{PY@tok@gi}{\def\PY@tc##1{\textcolor[rgb]{0.00,0.52,0.00}{##1}}}
\@namedef{PY@tok@gr}{\def\PY@tc##1{\textcolor[rgb]{0.89,0.00,0.00}{##1}}}
\@namedef{PY@tok@ge}{\let\PY@it=\textit}
\@namedef{PY@tok@gs}{\let\PY@bf=\textbf}
\@namedef{PY@tok@gp}{\let\PY@bf=\textbf\def\PY@tc##1{\textcolor[rgb]{0.00,0.00,0.50}{##1}}}
\@namedef{PY@tok@go}{\def\PY@tc##1{\textcolor[rgb]{0.44,0.44,0.44}{##1}}}
\@namedef{PY@tok@gt}{\def\PY@tc##1{\textcolor[rgb]{0.00,0.27,0.87}{##1}}}
\@namedef{PY@tok@err}{\def\PY@bc##1{{\setlength{\fboxsep}{\string -\fboxrule}\fcolorbox[rgb]{1.00,0.00,0.00}{1,1,1}{\strut ##1}}}}
\@namedef{PY@tok@kc}{\let\PY@bf=\textbf\def\PY@tc##1{\textcolor[rgb]{0.00,0.50,0.00}{##1}}}
\@namedef{PY@tok@kd}{\let\PY@bf=\textbf\def\PY@tc##1{\textcolor[rgb]{0.00,0.50,0.00}{##1}}}
\@namedef{PY@tok@kn}{\let\PY@bf=\textbf\def\PY@tc##1{\textcolor[rgb]{0.00,0.50,0.00}{##1}}}
\@namedef{PY@tok@kr}{\let\PY@bf=\textbf\def\PY@tc##1{\textcolor[rgb]{0.00,0.50,0.00}{##1}}}
\@namedef{PY@tok@bp}{\def\PY@tc##1{\textcolor[rgb]{0.00,0.50,0.00}{##1}}}
\@namedef{PY@tok@fm}{\def\PY@tc##1{\textcolor[rgb]{0.00,0.00,1.00}{##1}}}
\@namedef{PY@tok@vc}{\def\PY@tc##1{\textcolor[rgb]{0.10,0.09,0.49}{##1}}}
\@namedef{PY@tok@vg}{\def\PY@tc##1{\textcolor[rgb]{0.10,0.09,0.49}{##1}}}
\@namedef{PY@tok@vi}{\def\PY@tc##1{\textcolor[rgb]{0.10,0.09,0.49}{##1}}}
\@namedef{PY@tok@vm}{\def\PY@tc##1{\textcolor[rgb]{0.10,0.09,0.49}{##1}}}
\@namedef{PY@tok@sa}{\def\PY@tc##1{\textcolor[rgb]{0.73,0.13,0.13}{##1}}}
\@namedef{PY@tok@sb}{\def\PY@tc##1{\textcolor[rgb]{0.73,0.13,0.13}{##1}}}
\@namedef{PY@tok@sc}{\def\PY@tc##1{\textcolor[rgb]{0.73,0.13,0.13}{##1}}}
\@namedef{PY@tok@dl}{\def\PY@tc##1{\textcolor[rgb]{0.73,0.13,0.13}{##1}}}
\@namedef{PY@tok@s2}{\def\PY@tc##1{\textcolor[rgb]{0.73,0.13,0.13}{##1}}}
\@namedef{PY@tok@sh}{\def\PY@tc##1{\textcolor[rgb]{0.73,0.13,0.13}{##1}}}
\@namedef{PY@tok@s1}{\def\PY@tc##1{\textcolor[rgb]{0.73,0.13,0.13}{##1}}}
\@namedef{PY@tok@mb}{\def\PY@tc##1{\textcolor[rgb]{0.40,0.40,0.40}{##1}}}
\@namedef{PY@tok@mf}{\def\PY@tc##1{\textcolor[rgb]{0.40,0.40,0.40}{##1}}}
\@namedef{PY@tok@mh}{\def\PY@tc##1{\textcolor[rgb]{0.40,0.40,0.40}{##1}}}
\@namedef{PY@tok@mi}{\def\PY@tc##1{\textcolor[rgb]{0.40,0.40,0.40}{##1}}}
\@namedef{PY@tok@il}{\def\PY@tc##1{\textcolor[rgb]{0.40,0.40,0.40}{##1}}}
\@namedef{PY@tok@mo}{\def\PY@tc##1{\textcolor[rgb]{0.40,0.40,0.40}{##1}}}
\@namedef{PY@tok@ch}{\let\PY@it=\textit\def\PY@tc##1{\textcolor[rgb]{0.24,0.48,0.48}{##1}}}
\@namedef{PY@tok@cm}{\let\PY@it=\textit\def\PY@tc##1{\textcolor[rgb]{0.24,0.48,0.48}{##1}}}
\@namedef{PY@tok@cpf}{\let\PY@it=\textit\def\PY@tc##1{\textcolor[rgb]{0.24,0.48,0.48}{##1}}}
\@namedef{PY@tok@c1}{\let\PY@it=\textit\def\PY@tc##1{\textcolor[rgb]{0.24,0.48,0.48}{##1}}}
\@namedef{PY@tok@cs}{\let\PY@it=\textit\def\PY@tc##1{\textcolor[rgb]{0.24,0.48,0.48}{##1}}}

\def\PYZbs{\char`\\}
\def\PYZus{\char`\_}
\def\PYZob{\char`\{}
\def\PYZcb{\char`\}}
\def\PYZca{\char`\^}
\def\PYZam{\char`\&}
\def\PYZlt{\char`\<}
\def\PYZgt{\char`\>}
\def\PYZsh{\char`\#}
\def\PYZpc{\char`\%}
\def\PYZdl{\char`\$}
\def\PYZhy{\char`\-}
\def\PYZsq{\char`\'}
\def\PYZdq{\char`\"}
\def\PYZti{\char`\~}
% for compatibility with earlier versions
\def\PYZat{@}
\def\PYZlb{[}
\def\PYZrb{]}
\makeatother


    % For linebreaks inside Verbatim environment from package fancyvrb.
    \makeatletter
        \newbox\Wrappedcontinuationbox
        \newbox\Wrappedvisiblespacebox
        \newcommand*\Wrappedvisiblespace {\textcolor{red}{\textvisiblespace}}
        \newcommand*\Wrappedcontinuationsymbol {\textcolor{red}{\llap{\tiny$\m@th\hookrightarrow$}}}
        \newcommand*\Wrappedcontinuationindent {3ex }
        \newcommand*\Wrappedafterbreak {\kern\Wrappedcontinuationindent\copy\Wrappedcontinuationbox}
        % Take advantage of the already applied Pygments mark-up to insert
        % potential linebreaks for TeX processing.
        %        {, <, #, %, $, ' and ": go to next line.
        %        _, }, ^, &, >, - and ~: stay at end of broken line.
        % Use of \textquotesingle for straight quote.
        \newcommand*\Wrappedbreaksatspecials {%
            \def\PYGZus{\discretionary{\char`\_}{\Wrappedafterbreak}{\char`\_}}%
            \def\PYGZob{\discretionary{}{\Wrappedafterbreak\char`\{}{\char`\{}}%
            \def\PYGZcb{\discretionary{\char`\}}{\Wrappedafterbreak}{\char`\}}}%
            \def\PYGZca{\discretionary{\char`\^}{\Wrappedafterbreak}{\char`\^}}%
            \def\PYGZam{\discretionary{\char`\&}{\Wrappedafterbreak}{\char`\&}}%
            \def\PYGZlt{\discretionary{}{\Wrappedafterbreak\char`\<}{\char`\<}}%
            \def\PYGZgt{\discretionary{\char`\>}{\Wrappedafterbreak}{\char`\>}}%
            \def\PYGZsh{\discretionary{}{\Wrappedafterbreak\char`\#}{\char`\#}}%
            \def\PYGZpc{\discretionary{}{\Wrappedafterbreak\char`\%}{\char`\%}}%
            \def\PYGZdl{\discretionary{}{\Wrappedafterbreak\char`\$}{\char`\$}}%
            \def\PYGZhy{\discretionary{\char`\-}{\Wrappedafterbreak}{\char`\-}}%
            \def\PYGZsq{\discretionary{}{\Wrappedafterbreak\textquotesingle}{\textquotesingle}}%
            \def\PYGZdq{\discretionary{}{\Wrappedafterbreak\char`\"}{\char`\"}}%
            \def\PYGZti{\discretionary{\char`\~}{\Wrappedafterbreak}{\char`\~}}%
        }
        % Some characters . , ; ? ! / are not pygmentized.
        % This macro makes them "active" and they will insert potential linebreaks
        \newcommand*\Wrappedbreaksatpunct {%
            \lccode`\~`\.\lowercase{\def~}{\discretionary{\hbox{\char`\.}}{\Wrappedafterbreak}{\hbox{\char`\.}}}%
            \lccode`\~`\,\lowercase{\def~}{\discretionary{\hbox{\char`\,}}{\Wrappedafterbreak}{\hbox{\char`\,}}}%
            \lccode`\~`\;\lowercase{\def~}{\discretionary{\hbox{\char`\;}}{\Wrappedafterbreak}{\hbox{\char`\;}}}%
            \lccode`\~`\:\lowercase{\def~}{\discretionary{\hbox{\char`\:}}{\Wrappedafterbreak}{\hbox{\char`\:}}}%
            \lccode`\~`\?\lowercase{\def~}{\discretionary{\hbox{\char`\?}}{\Wrappedafterbreak}{\hbox{\char`\?}}}%
            \lccode`\~`\!\lowercase{\def~}{\discretionary{\hbox{\char`\!}}{\Wrappedafterbreak}{\hbox{\char`\!}}}%
            \lccode`\~`\/\lowercase{\def~}{\discretionary{\hbox{\char`\/}}{\Wrappedafterbreak}{\hbox{\char`\/}}}%
            \catcode`\.\active
            \catcode`\,\active
            \catcode`\;\active
            \catcode`\:\active
            \catcode`\?\active
            \catcode`\!\active
            \catcode`\/\active
            \lccode`\~`\~
        }
    \makeatother

    \let\OriginalVerbatim=\Verbatim
    \makeatletter
    \renewcommand{\Verbatim}[1][1]{%
        %\parskip\z@skip
        \sbox\Wrappedcontinuationbox {\Wrappedcontinuationsymbol}%
        \sbox\Wrappedvisiblespacebox {\FV@SetupFont\Wrappedvisiblespace}%
        \def\FancyVerbFormatLine ##1{\hsize\linewidth
            \vtop{\raggedright\hyphenpenalty\z@\exhyphenpenalty\z@
                \doublehyphendemerits\z@\finalhyphendemerits\z@
                \strut ##1\strut}%
        }%
        % If the linebreak is at a space, the latter will be displayed as visible
        % space at end of first line, and a continuation symbol starts next line.
        % Stretch/shrink are however usually zero for typewriter font.
        \def\FV@Space {%
            \nobreak\hskip\z@ plus\fontdimen3\font minus\fontdimen4\font
            \discretionary{\copy\Wrappedvisiblespacebox}{\Wrappedafterbreak}
            {\kern\fontdimen2\font}%
        }%

        % Allow breaks at special characters using \PYG... macros.
        \Wrappedbreaksatspecials
        % Breaks at punctuation characters . , ; ? ! and / need catcode=\active
        \OriginalVerbatim[#1,codes*=\Wrappedbreaksatpunct]%
    }
    \makeatother

    % Exact colors from NB
    \definecolor{incolor}{HTML}{303F9F}
    \definecolor{outcolor}{HTML}{D84315}
    \definecolor{cellborder}{HTML}{CFCFCF}
    \definecolor{cellbackground}{HTML}{F7F7F7}

    % prompt
    \makeatletter
    \newcommand{\boxspacing}{\kern\kvtcb@left@rule\kern\kvtcb@boxsep}
    \makeatother
    \newcommand{\prompt}[4]{
        {\ttfamily\llap{{\color{#2}[#3]:\hspace{3pt}#4}}\vspace{-\baselineskip}}
    }
    

    
    % Prevent overflowing lines due to hard-to-break entities
    \sloppy
    % Setup hyperref package
    \hypersetup{
      breaklinks=true,  % so long urls are correctly broken across lines
      colorlinks=true,
      urlcolor=urlcolor,
      linkcolor=linkcolor,
      citecolor=citecolor,
      }
    % Slightly bigger margins than the latex defaults
    
    \geometry{verbose,tmargin=1in,bmargin=1in,lmargin=1in,rmargin=1in}
    
    

\begin{document}
    
    \maketitle
    
    

    
    \begin{tcolorbox}[breakable, size=fbox, boxrule=1pt, pad at break*=1mm,colback=cellbackground, colframe=cellborder]
\prompt{In}{incolor}{88}{\boxspacing}
\begin{Verbatim}[commandchars=\\\{\}]
\PY{c+c1}{\PYZsh{} 基于深度学习的遥感影像分割}
\PY{k+kn}{import} \PY{n+nn}{numpy} \PY{k}{as} \PY{n+nn}{np}
\PY{k+kn}{import} \PY{n+nn}{matplotlib}\PY{n+nn}{.}\PY{n+nn}{pyplot} \PY{k}{as} \PY{n+nn}{plt}
\PY{k+kn}{import} \PY{n+nn}{tensorflow} \PY{k}{as} \PY{n+nn}{tf}
\PY{k+kn}{from} \PY{n+nn}{tensorflow} \PY{k+kn}{import} \PY{n}{keras}
\PY{k+kn}{from} \PY{n+nn}{tensorflow}\PY{n+nn}{.}\PY{n+nn}{keras} \PY{k+kn}{import} \PY{n}{layers}
\PY{k+kn}{import} \PY{n+nn}{os}
\PY{k+kn}{import} \PY{n+nn}{cv2}
\PY{k+kn}{import} \PY{n+nn}{numpy} \PY{k}{as} \PY{n+nn}{np}

\PY{c+c1}{\PYZsh{} 基于深度学习遥感影像分割实例分析}
\PY{c+c1}{\PYZsh{} 读取文件内的所以图像并将他们转为数组}
\PY{c+c1}{\PYZsh{} 指定包含图像的文件夹路径}
\PY{n}{folder\PYZus{}path} \PY{o}{=} \PY{l+s+s1}{\PYZsq{}}\PY{l+s+s1}{C:}\PY{l+s+s1}{\PYZbs{}}\PY{l+s+s1}{PHD}\PY{l+s+s1}{\PYZbs{}}\PY{l+s+s1}{deeplearning\PYZus{}2}\PY{l+s+s1}{\PYZbs{}}\PY{l+s+s1}{image\PYZus{}train}\PY{l+s+s1}{\PYZsq{}}

\PY{c+c1}{\PYZsh{} 初始化一个空列表来存储图像数组}
\PY{n}{image\PYZus{}data} \PY{o}{=} \PY{p}{[}\PY{p}{]}

\PY{c+c1}{\PYZsh{} 遍历文件夹中的所有图像文件}
\PY{k}{for} \PY{n}{filename} \PY{o+ow}{in} \PY{n}{os}\PY{o}{.}\PY{n}{listdir}\PY{p}{(}\PY{n}{folder\PYZus{}path}\PY{p}{)}\PY{p}{:}
    \PY{k}{if} \PY{n}{filename}\PY{o}{.}\PY{n}{endswith}\PY{p}{(}\PY{p}{(}\PY{l+s+s1}{\PYZsq{}}\PY{l+s+s1}{.jpg}\PY{l+s+s1}{\PYZsq{}}\PY{p}{,} \PY{l+s+s1}{\PYZsq{}}\PY{l+s+s1}{.jpeg}\PY{l+s+s1}{\PYZsq{}}\PY{p}{,} \PY{l+s+s1}{\PYZsq{}}\PY{l+s+s1}{.png}\PY{l+s+s1}{\PYZsq{}}\PY{p}{,}\PY{l+s+s2}{\PYZdq{}}\PY{l+s+s2}{.tif}\PY{l+s+s2}{\PYZdq{}}\PY{p}{)}\PY{p}{)}\PY{p}{:}  \PY{c+c1}{\PYZsh{} 仅处理特定图像文件类型}
        \PY{n}{file\PYZus{}path} \PY{o}{=} \PY{n}{os}\PY{o}{.}\PY{n}{path}\PY{o}{.}\PY{n}{join}\PY{p}{(}\PY{n}{folder\PYZus{}path}\PY{p}{,} \PY{n}{filename}\PY{p}{)}
        \PY{n}{image} \PY{o}{=} \PY{n}{cv2}\PY{o}{.}\PY{n}{imread}\PY{p}{(}\PY{n}{file\PYZus{}path}\PY{p}{)}  \PY{c+c1}{\PYZsh{} 使用OpenCV读取图像}
        \PY{k}{if} \PY{n}{image} \PY{o+ow}{is} \PY{o+ow}{not} \PY{k+kc}{None}\PY{p}{:}
            \PY{n}{image\PYZus{}data}\PY{o}{.}\PY{n}{append}\PY{p}{(}\PY{n}{image}\PY{p}{)}

\PY{c+c1}{\PYZsh{} 将图像数组转换为NumPy数组}
\PY{n}{image\PYZus{}data} \PY{o}{=} \PY{n}{np}\PY{o}{.}\PY{n}{array}\PY{p}{(}\PY{n}{image\PYZus{}data}\PY{p}{)}

\PY{c+c1}{\PYZsh{} 打印数组的形状,这将显示图像数量、高度、宽度和通道数}
\PY{n+nb}{print}\PY{p}{(}\PY{l+s+s2}{\PYZdq{}}\PY{l+s+s2}{Shape of image data array:}\PY{l+s+s2}{\PYZdq{}}\PY{p}{,} \PY{n}{image\PYZus{}data}\PY{o}{.}\PY{n}{shape}\PY{p}{)}
\end{Verbatim}
\end{tcolorbox}

    \begin{Verbatim}[commandchars=\\\{\}]
Shape of image data array: (176, 256, 256, 3)
    \end{Verbatim}

    \begin{tcolorbox}[breakable, size=fbox, boxrule=1pt, pad at break*=1mm,colback=cellbackground, colframe=cellborder]
\prompt{In}{incolor}{89}{\boxspacing}
\begin{Verbatim}[commandchars=\\\{\}]
\PY{n}{train\PYZus{}image} \PY{o}{=} \PY{n}{image\PYZus{}data}
\end{Verbatim}
\end{tcolorbox}

    \begin{tcolorbox}[breakable, size=fbox, boxrule=1pt, pad at break*=1mm,colback=cellbackground, colframe=cellborder]
\prompt{In}{incolor}{90}{\boxspacing}
\begin{Verbatim}[commandchars=\\\{\}]
\PY{c+c1}{\PYZsh{} 部分训练样本可视化}
\PY{n}{num\PYZus{}samples\PYZus{}to\PYZus{}visualize} \PY{o}{=} \PY{l+m+mi}{20}  \PY{c+c1}{\PYZsh{} 可视化前10个样本}
\PY{n}{plt}\PY{o}{.}\PY{n}{figure}\PY{p}{(}\PY{n}{figsize}\PY{o}{=}\PY{p}{(}\PY{l+m+mi}{15}\PY{p}{,} \PY{l+m+mi}{6}\PY{p}{)}\PY{p}{)}
\PY{k}{for} \PY{n}{i} \PY{o+ow}{in} \PY{n+nb}{range}\PY{p}{(}\PY{n}{num\PYZus{}samples\PYZus{}to\PYZus{}visualize}\PY{p}{)}\PY{p}{:}
    \PY{n}{plt}\PY{o}{.}\PY{n}{subplot}\PY{p}{(}\PY{l+m+mi}{4}\PY{p}{,} \PY{l+m+mi}{5}\PY{p}{,} \PY{n}{i} \PY{o}{+} \PY{l+m+mi}{1}\PY{p}{)}
    \PY{n}{plt}\PY{o}{.}\PY{n}{imshow}\PY{p}{(}\PY{n}{train\PYZus{}image}\PY{p}{[}\PY{n}{i}\PY{p}{]}\PY{p}{)}
    \PY{n}{plt}\PY{o}{.}\PY{n}{title}\PY{p}{(}\PY{l+s+sa}{f}\PY{l+s+s1}{\PYZsq{}}\PY{l+s+s1}{Sample }\PY{l+s+si}{\PYZob{}}\PY{n}{i}\PY{+w}{ }\PY{o}{+}\PY{+w}{ }\PY{l+m+mi}{1}\PY{l+s+si}{\PYZcb{}}\PY{l+s+s1}{\PYZsq{}}\PY{p}{)}
    \PY{n}{plt}\PY{o}{.}\PY{n}{axis}\PY{p}{(}\PY{l+s+s1}{\PYZsq{}}\PY{l+s+s1}{off}\PY{l+s+s1}{\PYZsq{}}\PY{p}{)}

\PY{n}{plt}\PY{o}{.}\PY{n}{show}\PY{p}{(}\PY{p}{)}
\end{Verbatim}
\end{tcolorbox}

    \begin{center}
    \adjustimage{max size={0.9\linewidth}{0.9\paperheight}}{output_2_0.png}
    \end{center}
    { \hspace*{\fill} \\}
    
    \begin{tcolorbox}[breakable, size=fbox, boxrule=1pt, pad at break*=1mm,colback=cellbackground, colframe=cellborder]
\prompt{In}{incolor}{ }{\boxspacing}
\begin{Verbatim}[commandchars=\\\{\}]

\end{Verbatim}
\end{tcolorbox}

    \begin{tcolorbox}[breakable, size=fbox, boxrule=1pt, pad at break*=1mm,colback=cellbackground, colframe=cellborder]
\prompt{In}{incolor}{ }{\boxspacing}
\begin{Verbatim}[commandchars=\\\{\}]
\PY{c+c1}{\PYZsh{} 一定要定位到你下载labelme时指定的文件夹,同时找到labelme\PYZus{}json\PYZus{}to\PYZus{}dataset.exe这个文件才能转成}
\PY{c+c1}{\PYZsh{} C:\PYZbs{}ProgramData\PYZbs{}anaconda3\PYZbs{}Scripts\PYZgt{}python labelme\PYZus{}json\PYZus{}to\PYZus{}dataset.exe C:/PHD/deeplearning\PYZus{}3/raw\PYZus{}label/original2.json}
\end{Verbatim}
\end{tcolorbox}

    \begin{tcolorbox}[breakable, size=fbox, boxrule=1pt, pad at break*=1mm,colback=cellbackground, colframe=cellborder]
\prompt{In}{incolor}{15}{\boxspacing}
\begin{Verbatim}[commandchars=\\\{\}]
\PY{c+c1}{\PYZsh{} 将png格式的标签文件分割成256*256大小的文件,这个文件是由labelme制作成的json文件转png格式文件}
\PY{k+kn}{from} \PY{n+nn}{PIL} \PY{k+kn}{import} \PY{n}{Image}
\PY{k+kn}{import} \PY{n+nn}{os}

\PY{c+c1}{\PYZsh{} 设置输入和输出目录}
\PY{n}{input\PYZus{}dir} \PY{o}{=} \PY{l+s+s1}{\PYZsq{}}\PY{l+s+s1}{C:/PHD/deeplearning\PYZus{}3/raw\PYZus{}label/label\PYZus{}image/}\PY{l+s+s1}{\PYZsq{}}  \PY{c+c1}{\PYZsh{} 输入目录,包含PNG图像,注意这个文件夹内只能保存一个png图像,不能有其他格式文件}
\PY{n}{output\PYZus{}dir} \PY{o}{=} \PY{l+s+s1}{\PYZsq{}}\PY{l+s+s1}{C:/PHD/deeplearning\PYZus{}3/cut\PYZus{}label}\PY{l+s+s1}{\PYZsq{}}  \PY{c+c1}{\PYZsh{} 输出目录,用于存储分割后的子图像}

\PY{c+c1}{\PYZsh{} 确保输出目录存在}
\PY{k}{if} \PY{o+ow}{not} \PY{n}{os}\PY{o}{.}\PY{n}{path}\PY{o}{.}\PY{n}{exists}\PY{p}{(}\PY{n}{output\PYZus{}dir}\PY{p}{)}\PY{p}{:}
    \PY{n}{os}\PY{o}{.}\PY{n}{makedirs}\PY{p}{(}\PY{n}{output\PYZus{}dir}\PY{p}{)}

\PY{c+c1}{\PYZsh{} 待分割的PNG图像文件列表}
\PY{n}{image\PYZus{}files} \PY{o}{=} \PY{n}{os}\PY{o}{.}\PY{n}{listdir}\PY{p}{(}\PY{n}{input\PYZus{}dir}\PY{p}{)}

\PY{c+c1}{\PYZsh{} 遍历每个PNG图像文件}
\PY{k}{for} \PY{n}{image\PYZus{}file} \PY{o+ow}{in} \PY{n}{image\PYZus{}files}\PY{p}{:}
    \PY{k}{if} \PY{n}{image\PYZus{}file}\PY{o}{.}\PY{n}{endswith}\PY{p}{(}\PY{l+s+s1}{\PYZsq{}}\PY{l+s+s1}{.png}\PY{l+s+s1}{\PYZsq{}}\PY{p}{)}\PY{p}{:}
        \PY{c+c1}{\PYZsh{} 打开PNG图像}
        \PY{n}{image\PYZus{}path} \PY{o}{=} \PY{n}{os}\PY{o}{.}\PY{n}{path}\PY{o}{.}\PY{n}{join}\PY{p}{(}\PY{n}{input\PYZus{}dir}\PY{p}{,} \PY{n}{image\PYZus{}file}\PY{p}{)}
        \PY{n}{img} \PY{o}{=} \PY{n}{Image}\PY{o}{.}\PY{n}{open}\PY{p}{(}\PY{n}{image\PYZus{}path}\PY{p}{)}
        
        \PY{c+c1}{\PYZsh{} 获取图像的宽度和高度}
        \PY{n}{width}\PY{p}{,} \PY{n}{height} \PY{o}{=} \PY{n}{img}\PY{o}{.}\PY{n}{size}
        
        \PY{c+c1}{\PYZsh{} 定义子图像大小}
        \PY{n}{sub\PYZus{}image\PYZus{}size} \PY{o}{=} \PY{l+m+mi}{256}
        
        \PY{c+c1}{\PYZsh{} 计算水平和垂直方向上的子图像数量}
        \PY{n}{num\PYZus{}horizontal\PYZus{}sub\PYZus{}images} \PY{o}{=} \PY{n}{width} \PY{o}{/}\PY{o}{/} \PY{n}{sub\PYZus{}image\PYZus{}size}
        \PY{n}{num\PYZus{}vertical\PYZus{}sub\PYZus{}images} \PY{o}{=} \PY{n}{height} \PY{o}{/}\PY{o}{/} \PY{n}{sub\PYZus{}image\PYZus{}size}
        
        \PY{c+c1}{\PYZsh{} 遍历子图像}
        \PY{k}{for} \PY{n}{i} \PY{o+ow}{in} \PY{n+nb}{range}\PY{p}{(}\PY{n}{num\PYZus{}horizontal\PYZus{}sub\PYZus{}images}\PY{p}{)}\PY{p}{:}
            \PY{k}{for} \PY{n}{j} \PY{o+ow}{in} \PY{n+nb}{range}\PY{p}{(}\PY{n}{num\PYZus{}vertical\PYZus{}sub\PYZus{}images}\PY{p}{)}\PY{p}{:}
                \PY{n}{left} \PY{o}{=} \PY{n}{i} \PY{o}{*} \PY{n}{sub\PYZus{}image\PYZus{}size}
                \PY{n}{upper} \PY{o}{=} \PY{n}{j} \PY{o}{*} \PY{n}{sub\PYZus{}image\PYZus{}size}
                \PY{n}{right} \PY{o}{=} \PY{p}{(}\PY{n}{i} \PY{o}{+} \PY{l+m+mi}{1}\PY{p}{)} \PY{o}{*} \PY{n}{sub\PYZus{}image\PYZus{}size}
                \PY{n}{lower} \PY{o}{=} \PY{p}{(}\PY{n}{j} \PY{o}{+} \PY{l+m+mi}{1}\PY{p}{)} \PY{o}{*} \PY{n}{sub\PYZus{}image\PYZus{}size}
                
                \PY{c+c1}{\PYZsh{} 裁剪子图像}
                \PY{n}{sub\PYZus{}img} \PY{o}{=} \PY{n}{img}\PY{o}{.}\PY{n}{crop}\PY{p}{(}\PY{p}{(}\PY{n}{left}\PY{p}{,} \PY{n}{upper}\PY{p}{,} \PY{n}{right}\PY{p}{,} \PY{n}{lower}\PY{p}{)}\PY{p}{)}
                
                \PY{c+c1}{\PYZsh{} 确保子图像的大小符合规则}
                \PY{k}{if} \PY{n}{sub\PYZus{}img}\PY{o}{.}\PY{n}{size} \PY{o}{==} \PY{p}{(}\PY{n}{sub\PYZus{}image\PYZus{}size}\PY{p}{,} \PY{n}{sub\PYZus{}image\PYZus{}size}\PY{p}{)}\PY{p}{:}
                    \PY{c+c1}{\PYZsh{} 保存符合规则的子图像到输出目录}
                    \PY{n}{output\PYZus{}file} \PY{o}{=} \PY{l+s+sa}{f}\PY{l+s+s2}{\PYZdq{}}\PY{l+s+si}{\PYZob{}}\PY{n}{image\PYZus{}file}\PY{p}{[}\PY{p}{:}\PY{o}{\PYZhy{}}\PY{l+m+mi}{4}\PY{p}{]}\PY{l+s+si}{\PYZcb{}}\PY{l+s+s2}{\PYZus{}}\PY{l+s+si}{\PYZob{}}\PY{n}{i}\PY{l+s+si}{\PYZcb{}}\PY{l+s+s2}{\PYZus{}}\PY{l+s+si}{\PYZob{}}\PY{n}{j}\PY{l+s+si}{\PYZcb{}}\PY{l+s+s2}{.png}\PY{l+s+s2}{\PYZdq{}}
                    \PY{n}{output\PYZus{}path} \PY{o}{=} \PY{n}{os}\PY{o}{.}\PY{n}{path}\PY{o}{.}\PY{n}{join}\PY{p}{(}\PY{n}{output\PYZus{}dir}\PY{p}{,} \PY{n}{output\PYZus{}file}\PY{p}{)}
                    \PY{n}{sub\PYZus{}img}\PY{o}{.}\PY{n}{save}\PY{p}{(}\PY{n}{output\PYZus{}path}\PY{p}{)}
                \PY{k}{else}\PY{p}{:}
                    \PY{c+c1}{\PYZsh{} 子图像大小不符合规则,不保存}
                    \PY{k}{pass}

\PY{c+c1}{\PYZsh{} 完成分割和删除不符合规则的子图像}
\PY{n+nb}{print}\PY{p}{(}\PY{l+s+s2}{\PYZdq{}}\PY{l+s+s2}{分割和删除不符合规则的子图像完成}\PY{l+s+s2}{\PYZdq{}}\PY{p}{)}
\end{Verbatim}
\end{tcolorbox}

    \begin{Verbatim}[commandchars=\\\{\}]
分割和删除不符合规则的子图像完成
    \end{Verbatim}

    \begin{tcolorbox}[breakable, size=fbox, boxrule=1pt, pad at break*=1mm,colback=cellbackground, colframe=cellborder]
\prompt{In}{incolor}{16}{\boxspacing}
\begin{Verbatim}[commandchars=\\\{\}]
\PY{c+c1}{\PYZsh{} 将tif栅格格式文件分割成256*256大小的文件,要确保原始图像与标签文件格式一致}
\PY{c+c1}{\PYZsh{} 输入图像路径}
\PY{n}{input\PYZus{}image\PYZus{}path} \PY{o}{=} \PY{l+s+s2}{\PYZdq{}}\PY{l+s+s2}{C:/PHD/deeplearning\PYZus{}3/raw\PYZus{}image/original\PYZus{}image.tif}\PY{l+s+s2}{\PYZdq{}}
\PY{c+c1}{\PYZsh{} 输出目录}
\PY{n}{output\PYZus{}dir} \PY{o}{=} \PY{l+s+s2}{\PYZdq{}}\PY{l+s+s2}{C:/PHD/deeplearning\PYZus{}3/cut\PYZus{}image}\PY{l+s+s2}{\PYZdq{}}
\PY{c+c1}{\PYZsh{} 裁剪后的图像大小}
\PY{n}{crop\PYZus{}width} \PY{o}{=} \PY{l+m+mi}{256}
\PY{n}{crop\PYZus{}height} \PY{o}{=} \PY{l+m+mi}{256}
\end{Verbatim}
\end{tcolorbox}

    \begin{tcolorbox}[breakable, size=fbox, boxrule=1pt, pad at break*=1mm,colback=cellbackground, colframe=cellborder]
\prompt{In}{incolor}{17}{\boxspacing}
\begin{Verbatim}[commandchars=\\\{\}]
\PY{c+c1}{\PYZsh{} 定义函数,用于裁剪tif格式文件}
\PY{k}{def} \PY{n+nf}{crop\PYZus{}image}\PY{p}{(}\PY{n}{input\PYZus{}image}\PY{p}{,} \PY{n}{output\PYZus{}dir}\PY{p}{,} \PY{n}{crop\PYZus{}width}\PY{p}{,} \PY{n}{crop\PYZus{}height}\PY{p}{)}\PY{p}{:}
    \PY{k}{if} \PY{o+ow}{not} \PY{n}{os}\PY{o}{.}\PY{n}{path}\PY{o}{.}\PY{n}{exists}\PY{p}{(}\PY{n}{output\PYZus{}dir}\PY{p}{)}\PY{p}{:}
        \PY{n}{os}\PY{o}{.}\PY{n}{makedirs}\PY{p}{(}\PY{n}{output\PYZus{}dir}\PY{p}{)}

    \PY{n}{img} \PY{o}{=} \PY{n}{Image}\PY{o}{.}\PY{n}{open}\PY{p}{(}\PY{n}{input\PYZus{}image}\PY{p}{)}
    \PY{n}{width}\PY{p}{,} \PY{n}{height} \PY{o}{=} \PY{n}{img}\PY{o}{.}\PY{n}{size}

    \PY{n}{num\PYZus{}rows} \PY{o}{=} \PY{n}{height} \PY{o}{/}\PY{o}{/} \PY{n}{crop\PYZus{}height}
    \PY{n}{num\PYZus{}cols} \PY{o}{=} \PY{n}{width} \PY{o}{/}\PY{o}{/} \PY{n}{crop\PYZus{}width}

    \PY{k}{for} \PY{n}{row} \PY{o+ow}{in} \PY{n+nb}{range}\PY{p}{(}\PY{n}{num\PYZus{}rows}\PY{p}{)}\PY{p}{:}
        \PY{k}{for} \PY{n}{col} \PY{o+ow}{in} \PY{n+nb}{range}\PY{p}{(}\PY{n}{num\PYZus{}cols}\PY{p}{)}\PY{p}{:}
            \PY{n}{left} \PY{o}{=} \PY{n}{col} \PY{o}{*} \PY{n}{crop\PYZus{}width}
            \PY{n}{top} \PY{o}{=} \PY{n}{row} \PY{o}{*} \PY{n}{crop\PYZus{}height}
            \PY{n}{right} \PY{o}{=} \PY{n}{left} \PY{o}{+} \PY{n}{crop\PYZus{}width}
            \PY{n}{bottom} \PY{o}{=} \PY{n}{top} \PY{o}{+} \PY{n}{crop\PYZus{}height}

            \PY{n}{cropped\PYZus{}img} \PY{o}{=} \PY{n}{img}\PY{o}{.}\PY{n}{crop}\PY{p}{(}\PY{p}{(}\PY{n}{left}\PY{p}{,} \PY{n}{top}\PY{p}{,} \PY{n}{right}\PY{p}{,} \PY{n}{bottom}\PY{p}{)}\PY{p}{)}

            \PY{c+c1}{\PYZsh{} 只保存尺寸正确的图像}
            \PY{k}{if} \PY{n}{cropped\PYZus{}img}\PY{o}{.}\PY{n}{size} \PY{o}{==} \PY{p}{(}\PY{n}{crop\PYZus{}width}\PY{p}{,} \PY{n}{crop\PYZus{}height}\PY{p}{)}\PY{p}{:}
                \PY{n}{output\PYZus{}path} \PY{o}{=} \PY{n}{os}\PY{o}{.}\PY{n}{path}\PY{o}{.}\PY{n}{join}\PY{p}{(}\PY{n}{output\PYZus{}dir}\PY{p}{,} \PY{l+s+sa}{f}\PY{l+s+s2}{\PYZdq{}}\PY{l+s+s2}{crop\PYZus{}}\PY{l+s+si}{\PYZob{}}\PY{n}{row}\PY{l+s+si}{\PYZcb{}}\PY{l+s+s2}{\PYZus{}}\PY{l+s+si}{\PYZob{}}\PY{n}{col}\PY{l+s+si}{\PYZcb{}}\PY{l+s+s2}{.tif}\PY{l+s+s2}{\PYZdq{}}\PY{p}{)}
                \PY{n}{cropped\PYZus{}img}\PY{o}{.}\PY{n}{save}\PY{p}{(}\PY{n}{output\PYZus{}path}\PY{p}{)}
            \PY{k}{else}\PY{p}{:}
                \PY{n+nb}{print}\PY{p}{(}\PY{l+s+sa}{f}\PY{l+s+s2}{\PYZdq{}}\PY{l+s+s2}{未能裁剪的图像:crop\PYZus{}}\PY{l+s+si}{\PYZob{}}\PY{n}{row}\PY{l+s+si}{\PYZcb{}}\PY{l+s+s2}{\PYZus{}}\PY{l+s+si}{\PYZob{}}\PY{n}{col}\PY{l+s+si}{\PYZcb{}}\PY{l+s+s2}{.tif 已删除}\PY{l+s+s2}{\PYZdq{}}\PY{p}{)}
                \PY{n}{os}\PY{o}{.}\PY{n}{remove}\PY{p}{(}\PY{n}{input\PYZus{}image}\PY{p}{)}

    \PY{n+nb}{print}\PY{p}{(}\PY{l+s+s2}{\PYZdq{}}\PY{l+s+s2}{裁剪完成}\PY{l+s+s2}{\PYZdq{}}\PY{p}{)}
\end{Verbatim}
\end{tcolorbox}

    \begin{tcolorbox}[breakable, size=fbox, boxrule=1pt, pad at break*=1mm,colback=cellbackground, colframe=cellborder]
\prompt{In}{incolor}{18}{\boxspacing}
\begin{Verbatim}[commandchars=\\\{\}]
\PY{k}{if} \PY{n+nv+vm}{\PYZus{}\PYZus{}name\PYZus{}\PYZus{}} \PY{o}{==} \PY{l+s+s2}{\PYZdq{}}\PY{l+s+s2}{\PYZus{}\PYZus{}main\PYZus{}\PYZus{}}\PY{l+s+s2}{\PYZdq{}}\PY{p}{:}
    \PY{n}{crop\PYZus{}image}\PY{p}{(}\PY{n}{input\PYZus{}image\PYZus{}path}\PY{p}{,} \PY{n}{output\PYZus{}dir}\PY{p}{,} \PY{n}{crop\PYZus{}width}\PY{p}{,} \PY{n}{crop\PYZus{}height}\PY{p}{)}
\end{Verbatim}
\end{tcolorbox}

    \begin{Verbatim}[commandchars=\\\{\}]
裁剪完成
    \end{Verbatim}

    \begin{tcolorbox}[breakable, size=fbox, boxrule=1pt, pad at break*=1mm,colback=cellbackground, colframe=cellborder]
\prompt{In}{incolor}{38}{\boxspacing}
\begin{Verbatim}[commandchars=\\\{\}]
\PY{c+c1}{\PYZsh{} 现在就可以将分割好的图像进行分配训练集和数据集,这一步可以打乱也可以按顺序自行分配}
\PY{c+c1}{\PYZsh{} 下面是将数据转为数组用于训练}
\PY{c+c1}{\PYZsh{} 读取文件内的所以图像并将他们转为数组(将原始图像进行转数组操作)}
\PY{k+kn}{import} \PY{n+nn}{os}
\PY{k+kn}{import} \PY{n+nn}{cv2}
\PY{k+kn}{import} \PY{n+nn}{numpy} \PY{k}{as} \PY{n+nn}{np}

\PY{c+c1}{\PYZsh{} 指定包含图像的文件夹路径}
\PY{n}{folder\PYZus{}path} \PY{o}{=} \PY{l+s+s1}{\PYZsq{}}\PY{l+s+s1}{C:}\PY{l+s+s1}{\PYZbs{}}\PY{l+s+s1}{PHD}\PY{l+s+s1}{\PYZbs{}}\PY{l+s+s1}{deeplearning\PYZus{}3}\PY{l+s+s1}{\PYZbs{}}\PY{l+s+s1}{model\PYZus{}data}\PY{l+s+s1}{\PYZbs{}}\PY{l+s+s1}{image\PYZus{}train}\PY{l+s+s1}{\PYZsq{}}

\PY{c+c1}{\PYZsh{} 初始化一个空列表来存储图像数组}
\PY{n}{image\PYZus{}data} \PY{o}{=} \PY{p}{[}\PY{p}{]}

\PY{c+c1}{\PYZsh{} 遍历文件夹中的所有图像文件}
\PY{k}{for} \PY{n}{filename} \PY{o+ow}{in} \PY{n}{os}\PY{o}{.}\PY{n}{listdir}\PY{p}{(}\PY{n}{folder\PYZus{}path}\PY{p}{)}\PY{p}{:}
    \PY{k}{if} \PY{n}{filename}\PY{o}{.}\PY{n}{endswith}\PY{p}{(}\PY{p}{(}\PY{l+s+s1}{\PYZsq{}}\PY{l+s+s1}{.jpg}\PY{l+s+s1}{\PYZsq{}}\PY{p}{,} \PY{l+s+s1}{\PYZsq{}}\PY{l+s+s1}{.jpeg}\PY{l+s+s1}{\PYZsq{}}\PY{p}{,} \PY{l+s+s1}{\PYZsq{}}\PY{l+s+s1}{.png}\PY{l+s+s1}{\PYZsq{}}\PY{p}{,}\PY{l+s+s2}{\PYZdq{}}\PY{l+s+s2}{.tif}\PY{l+s+s2}{\PYZdq{}}\PY{p}{)}\PY{p}{)}\PY{p}{:}  \PY{c+c1}{\PYZsh{} 仅处理特定图像文件类型}
        \PY{n}{file\PYZus{}path} \PY{o}{=} \PY{n}{os}\PY{o}{.}\PY{n}{path}\PY{o}{.}\PY{n}{join}\PY{p}{(}\PY{n}{folder\PYZus{}path}\PY{p}{,} \PY{n}{filename}\PY{p}{)}
        \PY{n}{image} \PY{o}{=} \PY{n}{cv2}\PY{o}{.}\PY{n}{imread}\PY{p}{(}\PY{n}{file\PYZus{}path}\PY{p}{)}  \PY{c+c1}{\PYZsh{} 使用OpenCV读取图像}
        \PY{k}{if} \PY{n}{image} \PY{o+ow}{is} \PY{o+ow}{not} \PY{k+kc}{None}\PY{p}{:}
            \PY{n}{image\PYZus{}data}\PY{o}{.}\PY{n}{append}\PY{p}{(}\PY{n}{image}\PY{p}{)}

\PY{c+c1}{\PYZsh{} 将图像数组转换为NumPy数组}
\PY{n}{image\PYZus{}data} \PY{o}{=} \PY{n}{np}\PY{o}{.}\PY{n}{array}\PY{p}{(}\PY{n}{image\PYZus{}data}\PY{p}{)}

\PY{c+c1}{\PYZsh{} 打印数组的形状,这将显示图像数量、高度、宽度和通道数}
\PY{n+nb}{print}\PY{p}{(}\PY{l+s+s2}{\PYZdq{}}\PY{l+s+s2}{Shape of image data array:}\PY{l+s+s2}{\PYZdq{}}\PY{p}{,} \PY{n}{image\PYZus{}data}\PY{o}{.}\PY{n}{shape}\PY{p}{)}
\end{Verbatim}
\end{tcolorbox}

    \begin{Verbatim}[commandchars=\\\{\}]
Shape of image data array: (184, 256, 256, 3)
    \end{Verbatim}

    \begin{tcolorbox}[breakable, size=fbox, boxrule=1pt, pad at break*=1mm,colback=cellbackground, colframe=cellborder]
\prompt{In}{incolor}{39}{\boxspacing}
\begin{Verbatim}[commandchars=\\\{\}]
\PY{n}{train\PYZus{}image} \PY{o}{=} \PY{n}{image\PYZus{}data}
\end{Verbatim}
\end{tcolorbox}

    \begin{tcolorbox}[breakable, size=fbox, boxrule=1pt, pad at break*=1mm,colback=cellbackground, colframe=cellborder]
\prompt{In}{incolor}{43}{\boxspacing}
\begin{Verbatim}[commandchars=\\\{\}]
\PY{c+c1}{\PYZsh{} 归一化,将所以像元数值都控制在【0,1】内}
\PY{n}{train\PYZus{}image\PYZus{}normal} \PY{o}{=} \PY{n}{train\PYZus{}image} \PY{o}{/} \PY{l+m+mf}{255.0}
\PY{n}{input\PYZus{}data} \PY{o}{=} \PY{n}{train\PYZus{}image\PYZus{}normal}
\end{Verbatim}
\end{tcolorbox}

    \begin{tcolorbox}[breakable, size=fbox, boxrule=1pt, pad at break*=1mm,colback=cellbackground, colframe=cellborder]
\prompt{In}{incolor}{41}{\boxspacing}
\begin{Verbatim}[commandchars=\\\{\}]
\PY{n}{num\PYZus{}samples\PYZus{}to\PYZus{}visualize} \PY{o}{=} \PY{l+m+mi}{20}  \PY{c+c1}{\PYZsh{} 可视化前20个样本}
\PY{n}{plt}\PY{o}{.}\PY{n}{figure}\PY{p}{(}\PY{n}{figsize}\PY{o}{=}\PY{p}{(}\PY{l+m+mi}{15}\PY{p}{,} \PY{l+m+mi}{6}\PY{p}{)}\PY{p}{)}
\PY{k}{for} \PY{n}{i} \PY{o+ow}{in} \PY{n+nb}{range}\PY{p}{(}\PY{n}{num\PYZus{}samples\PYZus{}to\PYZus{}visualize}\PY{p}{)}\PY{p}{:}
    \PY{n}{plt}\PY{o}{.}\PY{n}{subplot}\PY{p}{(}\PY{l+m+mi}{4}\PY{p}{,} \PY{l+m+mi}{5}\PY{p}{,} \PY{n}{i} \PY{o}{+} \PY{l+m+mi}{1}\PY{p}{)}
    \PY{n}{plt}\PY{o}{.}\PY{n}{imshow}\PY{p}{(}\PY{n}{input\PYZus{}data}\PY{p}{[}\PY{n}{i}\PY{p}{]}\PY{p}{)}
    \PY{n}{plt}\PY{o}{.}\PY{n}{title}\PY{p}{(}\PY{l+s+sa}{f}\PY{l+s+s1}{\PYZsq{}}\PY{l+s+s1}{Sample }\PY{l+s+si}{\PYZob{}}\PY{n}{i}\PY{+w}{ }\PY{o}{+}\PY{+w}{ }\PY{l+m+mi}{1}\PY{l+s+si}{\PYZcb{}}\PY{l+s+s1}{\PYZsq{}}\PY{p}{)}
    \PY{n}{plt}\PY{o}{.}\PY{n}{axis}\PY{p}{(}\PY{l+s+s1}{\PYZsq{}}\PY{l+s+s1}{off}\PY{l+s+s1}{\PYZsq{}}\PY{p}{)}

\PY{n}{plt}\PY{o}{.}\PY{n}{show}\PY{p}{(}\PY{p}{)}
\end{Verbatim}
\end{tcolorbox}

    \begin{center}
    \adjustimage{max size={0.9\linewidth}{0.9\paperheight}}{output_12_0.png}
    \end{center}
    { \hspace*{\fill} \\}
    
    \begin{tcolorbox}[breakable, size=fbox, boxrule=1pt, pad at break*=1mm,colback=cellbackground, colframe=cellborder]
\prompt{In}{incolor}{58}{\boxspacing}
\begin{Verbatim}[commandchars=\\\{\}]
\PY{c+c1}{\PYZsh{} 下面是将数据转为数组用于训练}
\PY{c+c1}{\PYZsh{} 读取文件内的所以图像并将他们转为数组(将标签图像进行转数组操作)}
\PY{k+kn}{import} \PY{n+nn}{os}
\PY{k+kn}{import} \PY{n+nn}{cv2}
\PY{k+kn}{import} \PY{n+nn}{numpy} \PY{k}{as} \PY{n+nn}{np}

\PY{c+c1}{\PYZsh{} 指定包含图像的文件夹路径}
\PY{n}{folder\PYZus{}path} \PY{o}{=} \PY{l+s+s1}{\PYZsq{}}\PY{l+s+s1}{C:}\PY{l+s+s1}{\PYZbs{}}\PY{l+s+s1}{PHD}\PY{l+s+s1}{\PYZbs{}}\PY{l+s+s1}{deeplearning\PYZus{}3}\PY{l+s+s1}{\PYZbs{}}\PY{l+s+s1}{model\PYZus{}data}\PY{l+s+s1}{\PYZbs{}}\PY{l+s+s1}{label\PYZus{}train}\PY{l+s+s1}{\PYZsq{}}

\PY{c+c1}{\PYZsh{} 初始化一个空列表来存储图像数组}
\PY{n}{image\PYZus{}data} \PY{o}{=} \PY{p}{[}\PY{p}{]}

\PY{c+c1}{\PYZsh{} 遍历文件夹中的所有图像文件}
\PY{k}{for} \PY{n}{filename} \PY{o+ow}{in} \PY{n}{os}\PY{o}{.}\PY{n}{listdir}\PY{p}{(}\PY{n}{folder\PYZus{}path}\PY{p}{)}\PY{p}{:}
    \PY{k}{if} \PY{n}{filename}\PY{o}{.}\PY{n}{endswith}\PY{p}{(}\PY{p}{(}\PY{l+s+s1}{\PYZsq{}}\PY{l+s+s1}{.jpg}\PY{l+s+s1}{\PYZsq{}}\PY{p}{,} \PY{l+s+s1}{\PYZsq{}}\PY{l+s+s1}{.jpeg}\PY{l+s+s1}{\PYZsq{}}\PY{p}{,} \PY{l+s+s1}{\PYZsq{}}\PY{l+s+s1}{.png}\PY{l+s+s1}{\PYZsq{}}\PY{p}{,}\PY{l+s+s2}{\PYZdq{}}\PY{l+s+s2}{.tif}\PY{l+s+s2}{\PYZdq{}}\PY{p}{)}\PY{p}{)}\PY{p}{:}  \PY{c+c1}{\PYZsh{} 仅处理特定图像文件类型}
        \PY{n}{file\PYZus{}path} \PY{o}{=} \PY{n}{os}\PY{o}{.}\PY{n}{path}\PY{o}{.}\PY{n}{join}\PY{p}{(}\PY{n}{folder\PYZus{}path}\PY{p}{,} \PY{n}{filename}\PY{p}{)}
        \PY{n}{image} \PY{o}{=} \PY{n}{cv2}\PY{o}{.}\PY{n}{imread}\PY{p}{(}\PY{n}{file\PYZus{}path}\PY{p}{)}  \PY{c+c1}{\PYZsh{} 使用OpenCV读取图像}
        \PY{k}{if} \PY{n}{image} \PY{o+ow}{is} \PY{o+ow}{not} \PY{k+kc}{None}\PY{p}{:}
            \PY{n}{image\PYZus{}data}\PY{o}{.}\PY{n}{append}\PY{p}{(}\PY{n}{image}\PY{p}{)}

\PY{c+c1}{\PYZsh{} 将图像数组转换为NumPy数组}
\PY{n}{image\PYZus{}data} \PY{o}{=} \PY{n}{np}\PY{o}{.}\PY{n}{array}\PY{p}{(}\PY{n}{image\PYZus{}data}\PY{p}{)}

\PY{c+c1}{\PYZsh{} 打印数组的形状,这将显示图像数量、高度、宽度和通道数}
\PY{n+nb}{print}\PY{p}{(}\PY{l+s+s2}{\PYZdq{}}\PY{l+s+s2}{Shape of image data array:}\PY{l+s+s2}{\PYZdq{}}\PY{p}{,} \PY{n}{image\PYZus{}data}\PY{o}{.}\PY{n}{shape}\PY{p}{)}
\end{Verbatim}
\end{tcolorbox}

    \begin{Verbatim}[commandchars=\\\{\}]
Shape of image data array: (184, 256, 256, 3)
    \end{Verbatim}

    \begin{tcolorbox}[breakable, size=fbox, boxrule=1pt, pad at break*=1mm,colback=cellbackground, colframe=cellborder]
\prompt{In}{incolor}{ }{\boxspacing}
\begin{Verbatim}[commandchars=\\\{\}]

\end{Verbatim}
\end{tcolorbox}

    \begin{tcolorbox}[breakable, size=fbox, boxrule=1pt, pad at break*=1mm,colback=cellbackground, colframe=cellborder]
\prompt{In}{incolor}{59}{\boxspacing}
\begin{Verbatim}[commandchars=\\\{\}]
\PY{n}{label\PYZus{}image} \PY{o}{=} \PY{n}{image\PYZus{}data}
\PY{n}{labels} \PY{o}{=} \PY{n}{label\PYZus{}image}
\PY{n}{labels}
\end{Verbatim}
\end{tcolorbox}

            \begin{tcolorbox}[breakable, size=fbox, boxrule=.5pt, pad at break*=1mm, opacityfill=0]
\prompt{Out}{outcolor}{59}{\boxspacing}
\begin{Verbatim}[commandchars=\\\{\}]
array([[[[  0,   0, 128],
         [  0,   0, 128],
         [  0,   0, 128],
         {\ldots},
         [  0,   0, 128],
         [  0,   0, 128],
         [  0,   0, 128]],

        [[  0,   0, 128],
         [  0,   0, 128],
         [  0,   0, 128],
         {\ldots},
         [  0,   0, 128],
         [  0,   0, 128],
         [  0,   0, 128]],

        [[  0,   0, 128],
         [  0,   0, 128],
         [  0,   0, 128],
         {\ldots},
         [  0,   0, 128],
         [  0,   0, 128],
         [  0,   0, 128]],

        {\ldots},

        [[  0,   0,   0],
         [  0,   0,   0],
         [  0,   0,   0],
         {\ldots},
         [  0,   0, 128],
         [  0,   0, 128],
         [  0,   0, 128]],

        [[  0,   0,   0],
         [  0,   0,   0],
         [  0,   0,   0],
         {\ldots},
         [  0,   0, 128],
         [  0,   0, 128],
         [  0,   0, 128]],

        [[  0,   0,   0],
         [  0,   0,   0],
         [  0,   0,   0],
         {\ldots},
         [  0,   0, 128],
         [  0,   0, 128],
         [  0,   0, 128]]],


       [[[  0,   0,   0],
         [  0,   0,   0],
         [  0,   0,   0],
         {\ldots},
         [  0,   0, 128],
         [  0,   0, 128],
         [  0,   0, 128]],

        [[  0,   0,   0],
         [  0,   0,   0],
         [  0,   0,   0],
         {\ldots},
         [  0,   0, 128],
         [  0,   0, 128],
         [  0,   0, 128]],

        [[  0,   0,   0],
         [  0,   0,   0],
         [  0,   0,   0],
         {\ldots},
         [  0,   0, 128],
         [  0,   0, 128],
         [  0,   0, 128]],

        {\ldots},

        [[  0,   0,   0],
         [  0,   0,   0],
         [  0,   0,   0],
         {\ldots},
         [  0,   0,   0],
         [  0,   0,   0],
         [  0,   0,   0]],

        [[  0,   0,   0],
         [  0,   0,   0],
         [  0,   0,   0],
         {\ldots},
         [  0,   0,   0],
         [  0,   0,   0],
         [  0,   0,   0]],

        [[  0,   0,   0],
         [  0,   0,   0],
         [  0,   0,   0],
         {\ldots},
         [  0,   0,   0],
         [  0,   0,   0],
         [  0,   0,   0]]],


       [[[  0,   0,   0],
         [  0,   0,   0],
         [  0,   0,   0],
         {\ldots},
         [  0,   0,   0],
         [  0,   0,   0],
         [  0,   0,   0]],

        [[  0,   0,   0],
         [  0,   0,   0],
         [  0,   0,   0],
         {\ldots},
         [  0,   0,   0],
         [  0,   0,   0],
         [  0,   0,   0]],

        [[  0,   0,   0],
         [  0,   0,   0],
         [  0,   0,   0],
         {\ldots},
         [  0,   0,   0],
         [  0,   0,   0],
         [  0,   0,   0]],

        {\ldots},

        [[  0,   0,   0],
         [  0,   0,   0],
         [  0,   0,   0],
         {\ldots},
         [  0,   0,   0],
         [  0,   0,   0],
         [  0,   0,   0]],

        [[  0,   0,   0],
         [  0,   0,   0],
         [  0,   0,   0],
         {\ldots},
         [  0,   0,   0],
         [  0,   0,   0],
         [  0,   0,   0]],

        [[  0,   0,   0],
         [  0,   0,   0],
         [  0,   0,   0],
         {\ldots},
         [  0,   0,   0],
         [  0,   0,   0],
         [  0,   0,   0]]],


       {\ldots},


       [[[  0,   0, 128],
         [  0,   0, 128],
         [  0,   0, 128],
         {\ldots},
         [  0,   0, 128],
         [  0,   0, 128],
         [  0,   0, 128]],

        [[  0,   0, 128],
         [  0,   0, 128],
         [  0,   0, 128],
         {\ldots},
         [  0,   0, 128],
         [  0,   0, 128],
         [  0,   0, 128]],

        [[  0,   0, 128],
         [  0,   0, 128],
         [  0,   0, 128],
         {\ldots},
         [  0,   0, 128],
         [  0,   0, 128],
         [  0,   0, 128]],

        {\ldots},

        [[  0,   0, 128],
         [  0,   0, 128],
         [  0,   0, 128],
         {\ldots},
         [  0,   0, 128],
         [  0,   0, 128],
         [  0,   0, 128]],

        [[  0,   0, 128],
         [  0,   0, 128],
         [  0,   0, 128],
         {\ldots},
         [  0,   0, 128],
         [  0,   0, 128],
         [  0,   0, 128]],

        [[  0,   0, 128],
         [  0,   0, 128],
         [  0,   0, 128],
         {\ldots},
         [  0,   0, 128],
         [  0,   0, 128],
         [  0,   0, 128]]],


       [[[  0,   0, 128],
         [  0,   0, 128],
         [  0,   0, 128],
         {\ldots},
         [  0,   0, 128],
         [  0,   0, 128],
         [  0,   0, 128]],

        [[  0,   0, 128],
         [  0,   0, 128],
         [  0,   0, 128],
         {\ldots},
         [  0,   0, 128],
         [  0,   0, 128],
         [  0,   0, 128]],

        [[  0,   0, 128],
         [  0,   0, 128],
         [  0,   0, 128],
         {\ldots},
         [  0,   0, 128],
         [  0,   0, 128],
         [  0,   0, 128]],

        {\ldots},

        [[  0,   0, 128],
         [  0,   0, 128],
         [  0,   0, 128],
         {\ldots},
         [  0,   0, 128],
         [  0,   0, 128],
         [  0,   0, 128]],

        [[  0,   0, 128],
         [  0,   0, 128],
         [  0,   0, 128],
         {\ldots},
         [  0,   0, 128],
         [  0,   0, 128],
         [  0,   0, 128]],

        [[  0,   0, 128],
         [  0,   0, 128],
         [  0,   0, 128],
         {\ldots},
         [  0,   0, 128],
         [  0,   0, 128],
         [  0,   0, 128]]],


       [[[  0,   0, 128],
         [  0,   0, 128],
         [  0,   0, 128],
         {\ldots},
         [  0,   0, 128],
         [  0,   0, 128],
         [  0,   0, 128]],

        [[  0,   0, 128],
         [  0,   0, 128],
         [  0,   0, 128],
         {\ldots},
         [  0,   0, 128],
         [  0,   0, 128],
         [  0,   0, 128]],

        [[  0,   0, 128],
         [  0,   0, 128],
         [  0,   0, 128],
         {\ldots},
         [  0,   0, 128],
         [  0,   0, 128],
         [  0,   0, 128]],

        {\ldots},

        [[  0,   0, 128],
         [  0,   0, 128],
         [  0,   0, 128],
         {\ldots},
         [  0,   0, 128],
         [  0,   0, 128],
         [  0,   0, 128]],

        [[  0,   0, 128],
         [  0,   0, 128],
         [  0,   0, 128],
         {\ldots},
         [  0,   0, 128],
         [  0,   0, 128],
         [  0,   0, 128]],

        [[  0,   0, 128],
         [  0,   0, 128],
         [  0,   0, 128],
         {\ldots},
         [  0,   0, 128],
         [  0,   0, 128],
         [  0,   0, 128]]]], dtype=uint8)
\end{Verbatim}
\end{tcolorbox}
        
    \begin{tcolorbox}[breakable, size=fbox, boxrule=1pt, pad at break*=1mm,colback=cellbackground, colframe=cellborder]
\prompt{In}{incolor}{63}{\boxspacing}
\begin{Verbatim}[commandchars=\\\{\}]
\PY{c+c1}{\PYZsh{} 将(184, 256, 256, 3)形状的标签数组转为(184, 256, 256, 1)的数组,主要是将【0,0,128】转为1,将【0.0.0】转为0.}
\PY{c+c1}{\PYZsh{}在这个示例中,我们首先创建了一个形状为 (184, 256, 256, 3) 的示例四维数组 array\PYZus{}4d。然后,我们定义了要替换的值}
\PY{c+c1}{\PYZsh{}replace\PYZus{}values 和替代值 replacement\PYZus{}value。接下来,我们创建一个新的四维数组 new\PYZus{}array\PYZus{}4d,其形状与输入数组相同,}
\PY{c+c1}{\PYZsh{}但初始值为0。然后,我们使用 np.all 函数创建了两个布尔掩码 mask\PYZus{}replace 和 mask\PYZus{}zero,分别用于标识要替换的值和要替换}
\PY{c+c1}{\PYZsh{}为零的值的位置。最后,我们根据这些掩码填充新数组,并将其维度从 (184, 256, 256) 扩展为 (184, 256, 256, 1)。}

\PY{c+c1}{\PYZsh{}请确保用你的实际数组替换示例中的 array\PYZus{}4d,并根据需要自定义要替换的值和替代值。这个代码段演示了如何将数组转换并进行值替换。}

\PY{n}{replace\PYZus{}values} \PY{o}{=} \PY{n}{np}\PY{o}{.}\PY{n}{array}\PY{p}{(}\PY{p}{[}\PY{l+m+mi}{0}\PY{p}{,} \PY{l+m+mi}{0}\PY{p}{,} \PY{l+m+mi}{128}\PY{p}{]}\PY{p}{)}
\PY{n}{replacement\PYZus{}value} \PY{o}{=} \PY{n}{np}\PY{o}{.}\PY{n}{array}\PY{p}{(}\PY{p}{[}\PY{l+m+mi}{1}\PY{p}{]}\PY{p}{)}

\PY{c+c1}{\PYZsh{} 创建一个与输入数组相同形状的新数组,初始值为0}
\PY{n}{new\PYZus{}array\PYZus{}4d} \PY{o}{=} \PY{n}{np}\PY{o}{.}\PY{n}{zeros\PYZus{}like}\PY{p}{(}\PY{n}{labels}\PY{p}{[}\PY{p}{:}\PY{p}{,} \PY{p}{:}\PY{p}{,} \PY{p}{:}\PY{p}{,} \PY{l+m+mi}{0}\PY{p}{]}\PY{p}{,} \PY{n}{dtype}\PY{o}{=}\PY{n}{np}\PY{o}{.}\PY{n}{uint8}\PY{p}{)}

\PY{c+c1}{\PYZsh{} 使用 np.all 函数找到要替换的值}
\PY{n}{mask\PYZus{}replace} \PY{o}{=} \PY{n}{np}\PY{o}{.}\PY{n}{all}\PY{p}{(}\PY{n}{labels} \PY{o}{==} \PY{n}{replace\PYZus{}values}\PY{p}{,} \PY{n}{axis}\PY{o}{=}\PY{o}{\PYZhy{}}\PY{l+m+mi}{1}\PY{p}{)}

\PY{c+c1}{\PYZsh{} 使用 np.all 函数找到要替换的值}
\PY{n}{mask\PYZus{}zero} \PY{o}{=} \PY{n}{np}\PY{o}{.}\PY{n}{all}\PY{p}{(}\PY{n}{labels} \PY{o}{==} \PY{p}{[}\PY{l+m+mi}{0}\PY{p}{,} \PY{l+m+mi}{0}\PY{p}{,} \PY{l+m+mi}{0}\PY{p}{]}\PY{p}{,} \PY{n}{axis}\PY{o}{=}\PY{o}{\PYZhy{}}\PY{l+m+mi}{1}\PY{p}{)}

\PY{c+c1}{\PYZsh{} 根据替代值和替换值填充新数组}
\PY{n}{new\PYZus{}array\PYZus{}4d}\PY{p}{[}\PY{n}{mask\PYZus{}replace}\PY{p}{]} \PY{o}{=} \PY{n}{replacement\PYZus{}value}
\PY{n}{new\PYZus{}array\PYZus{}4d}\PY{p}{[}\PY{n}{mask\PYZus{}zero}\PY{p}{]} \PY{o}{=} \PY{l+m+mi}{0}  \PY{c+c1}{\PYZsh{} 将 [0, 0, 0] 替换为 [0]}

\PY{c+c1}{\PYZsh{} 将新数组的维度从 (184, 256, 256) 扩展为 (184, 256, 256, 1)}
\PY{n}{new\PYZus{}array\PYZus{}4d} \PY{o}{=} \PY{n}{new\PYZus{}array\PYZus{}4d}\PY{p}{[}\PY{o}{.}\PY{o}{.}\PY{o}{.}\PY{p}{,} \PY{n}{np}\PY{o}{.}\PY{n}{newaxis}\PY{p}{]}

\PY{c+c1}{\PYZsh{} 输出结果}
\PY{n+nb}{print}\PY{p}{(}\PY{n}{new\PYZus{}array\PYZus{}4d}\PY{o}{.}\PY{n}{shape}\PY{p}{)}
\end{Verbatim}
\end{tcolorbox}

    \begin{Verbatim}[commandchars=\\\{\}]
(184, 256, 256, 1)
    \end{Verbatim}

    \begin{tcolorbox}[breakable, size=fbox, boxrule=1pt, pad at break*=1mm,colback=cellbackground, colframe=cellborder]
\prompt{In}{incolor}{65}{\boxspacing}
\begin{Verbatim}[commandchars=\\\{\}]
\PY{n}{labels} \PY{o}{=} \PY{n}{new\PYZus{}array\PYZus{}4d}
\end{Verbatim}
\end{tcolorbox}

    \begin{tcolorbox}[breakable, size=fbox, boxrule=1pt, pad at break*=1mm,colback=cellbackground, colframe=cellborder]
\prompt{In}{incolor}{73}{\boxspacing}
\begin{Verbatim}[commandchars=\\\{\}]
\PY{n}{num\PYZus{}samples\PYZus{}to\PYZus{}visualize} \PY{o}{=} \PY{l+m+mi}{20}  \PY{c+c1}{\PYZsh{} 可视化前20个样本}
\PY{n}{plt}\PY{o}{.}\PY{n}{figure}\PY{p}{(}\PY{n}{figsize}\PY{o}{=}\PY{p}{(}\PY{l+m+mi}{15}\PY{p}{,} \PY{l+m+mi}{6}\PY{p}{)}\PY{p}{)}
\PY{k}{for} \PY{n}{i} \PY{o+ow}{in} \PY{n+nb}{range}\PY{p}{(}\PY{n}{num\PYZus{}samples\PYZus{}to\PYZus{}visualize}\PY{p}{)}\PY{p}{:}
    \PY{n}{plt}\PY{o}{.}\PY{n}{subplot}\PY{p}{(}\PY{l+m+mi}{4}\PY{p}{,} \PY{l+m+mi}{5}\PY{p}{,} \PY{n}{i} \PY{o}{+} \PY{l+m+mi}{1}\PY{p}{)}
    \PY{n}{plt}\PY{o}{.}\PY{n}{imshow}\PY{p}{(}\PY{n}{labels}\PY{p}{[}\PY{n}{i}\PY{p}{]}\PY{p}{)}
    \PY{n}{plt}\PY{o}{.}\PY{n}{title}\PY{p}{(}\PY{l+s+sa}{f}\PY{l+s+s1}{\PYZsq{}}\PY{l+s+s1}{Sample }\PY{l+s+si}{\PYZob{}}\PY{n}{i}\PY{+w}{ }\PY{o}{+}\PY{+w}{ }\PY{l+m+mi}{1}\PY{l+s+si}{\PYZcb{}}\PY{l+s+s1}{\PYZsq{}}\PY{p}{)}
    \PY{n}{plt}\PY{o}{.}\PY{n}{axis}\PY{p}{(}\PY{l+s+s1}{\PYZsq{}}\PY{l+s+s1}{off}\PY{l+s+s1}{\PYZsq{}}\PY{p}{)}

\PY{n}{plt}\PY{o}{.}\PY{n}{show}\PY{p}{(}\PY{p}{)}
\end{Verbatim}
\end{tcolorbox}

    \begin{center}
    \adjustimage{max size={0.9\linewidth}{0.9\paperheight}}{output_18_0.png}
    \end{center}
    { \hspace*{\fill} \\}
    
    \begin{tcolorbox}[breakable, size=fbox, boxrule=1pt, pad at break*=1mm,colback=cellbackground, colframe=cellborder]
\prompt{In}{incolor}{ }{\boxspacing}
\begin{Verbatim}[commandchars=\\\{\}]

\end{Verbatim}
\end{tcolorbox}

    \begin{tcolorbox}[breakable, size=fbox, boxrule=1pt, pad at break*=1mm,colback=cellbackground, colframe=cellborder]
\prompt{In}{incolor}{67}{\boxspacing}
\begin{Verbatim}[commandchars=\\\{\}]
\PY{n}{num\PYZus{}samples\PYZus{}to\PYZus{}visualize} \PY{o}{=} \PY{l+m+mi}{20}  \PY{c+c1}{\PYZsh{} 可视化前20个样本}
\PY{n}{plt}\PY{o}{.}\PY{n}{figure}\PY{p}{(}\PY{n}{figsize}\PY{o}{=}\PY{p}{(}\PY{l+m+mi}{15}\PY{p}{,} \PY{l+m+mi}{6}\PY{p}{)}\PY{p}{)}
\PY{k}{for} \PY{n}{i} \PY{o+ow}{in} \PY{n+nb}{range}\PY{p}{(}\PY{n}{num\PYZus{}samples\PYZus{}to\PYZus{}visualize}\PY{p}{)}\PY{p}{:}
    \PY{n}{plt}\PY{o}{.}\PY{n}{subplot}\PY{p}{(}\PY{l+m+mi}{4}\PY{p}{,} \PY{l+m+mi}{5}\PY{p}{,} \PY{n}{i} \PY{o}{+} \PY{l+m+mi}{1}\PY{p}{)}
    \PY{n}{plt}\PY{o}{.}\PY{n}{imshow}\PY{p}{(}\PY{n}{label\PYZus{}image}\PY{p}{[}\PY{n}{i}\PY{p}{]}\PY{p}{)}
    \PY{n}{plt}\PY{o}{.}\PY{n}{title}\PY{p}{(}\PY{l+s+sa}{f}\PY{l+s+s1}{\PYZsq{}}\PY{l+s+s1}{Sample }\PY{l+s+si}{\PYZob{}}\PY{n}{i}\PY{+w}{ }\PY{o}{+}\PY{+w}{ }\PY{l+m+mi}{1}\PY{l+s+si}{\PYZcb{}}\PY{l+s+s1}{\PYZsq{}}\PY{p}{)}
    \PY{n}{plt}\PY{o}{.}\PY{n}{axis}\PY{p}{(}\PY{l+s+s1}{\PYZsq{}}\PY{l+s+s1}{off}\PY{l+s+s1}{\PYZsq{}}\PY{p}{)}

\PY{n}{plt}\PY{o}{.}\PY{n}{show}\PY{p}{(}\PY{p}{)}
\end{Verbatim}
\end{tcolorbox}

    \begin{center}
    \adjustimage{max size={0.9\linewidth}{0.9\paperheight}}{output_20_0.png}
    \end{center}
    { \hspace*{\fill} \\}
    
    \begin{tcolorbox}[breakable, size=fbox, boxrule=1pt, pad at break*=1mm,colback=cellbackground, colframe=cellborder]
\prompt{In}{incolor}{ }{\boxspacing}
\begin{Verbatim}[commandchars=\\\{\}]

\end{Verbatim}
\end{tcolorbox}

    \begin{tcolorbox}[breakable, size=fbox, boxrule=1pt, pad at break*=1mm,colback=cellbackground, colframe=cellborder]
\prompt{In}{incolor}{74}{\boxspacing}
\begin{Verbatim}[commandchars=\\\{\}]
\PY{c+c1}{\PYZsh{} 创建一个简单的图像分割模型}
\PY{n}{model} \PY{o}{=} \PY{n}{keras}\PY{o}{.}\PY{n}{Sequential}\PY{p}{(}\PY{p}{[}
    \PY{n}{layers}\PY{o}{.}\PY{n}{Input}\PY{p}{(}\PY{n}{shape}\PY{o}{=}\PY{p}{(}\PY{l+m+mi}{256}\PY{p}{,} \PY{l+m+mi}{256}\PY{p}{,} \PY{l+m+mi}{3}\PY{p}{)}\PY{p}{)}\PY{p}{,}  \PY{c+c1}{\PYZsh{} 输入图像大小}
    \PY{n}{layers}\PY{o}{.}\PY{n}{Conv2D}\PY{p}{(}\PY{l+m+mi}{32}\PY{p}{,} \PY{p}{(}\PY{l+m+mi}{3}\PY{p}{,} \PY{l+m+mi}{3}\PY{p}{)}\PY{p}{,} \PY{n}{activation}\PY{o}{=}\PY{l+s+s1}{\PYZsq{}}\PY{l+s+s1}{relu}\PY{l+s+s1}{\PYZsq{}}\PY{p}{,} \PY{n}{padding}\PY{o}{=}\PY{l+s+s1}{\PYZsq{}}\PY{l+s+s1}{same}\PY{l+s+s1}{\PYZsq{}}\PY{p}{)}\PY{p}{,}
    \PY{n}{layers}\PY{o}{.}\PY{n}{MaxPooling2D}\PY{p}{(}\PY{p}{(}\PY{l+m+mi}{2}\PY{p}{,} \PY{l+m+mi}{2}\PY{p}{)}\PY{p}{)}\PY{p}{,}
    \PY{n}{layers}\PY{o}{.}\PY{n}{Conv2D}\PY{p}{(}\PY{l+m+mi}{64}\PY{p}{,} \PY{p}{(}\PY{l+m+mi}{3}\PY{p}{,} \PY{l+m+mi}{3}\PY{p}{)}\PY{p}{,} \PY{n}{activation}\PY{o}{=}\PY{l+s+s1}{\PYZsq{}}\PY{l+s+s1}{relu}\PY{l+s+s1}{\PYZsq{}}\PY{p}{,} \PY{n}{padding}\PY{o}{=}\PY{l+s+s1}{\PYZsq{}}\PY{l+s+s1}{same}\PY{l+s+s1}{\PYZsq{}}\PY{p}{)}\PY{p}{,}
    \PY{n}{layers}\PY{o}{.}\PY{n}{MaxPooling2D}\PY{p}{(}\PY{p}{(}\PY{l+m+mi}{2}\PY{p}{,} \PY{l+m+mi}{2}\PY{p}{)}\PY{p}{)}\PY{p}{,}
    \PY{n}{layers}\PY{o}{.}\PY{n}{Conv2D}\PY{p}{(}\PY{l+m+mi}{64}\PY{p}{,} \PY{p}{(}\PY{l+m+mi}{3}\PY{p}{,} \PY{l+m+mi}{3}\PY{p}{)}\PY{p}{,} \PY{n}{activation}\PY{o}{=}\PY{l+s+s1}{\PYZsq{}}\PY{l+s+s1}{relu}\PY{l+s+s1}{\PYZsq{}}\PY{p}{,} \PY{n}{padding}\PY{o}{=}\PY{l+s+s1}{\PYZsq{}}\PY{l+s+s1}{same}\PY{l+s+s1}{\PYZsq{}}\PY{p}{)}\PY{p}{,}
    \PY{n}{layers}\PY{o}{.}\PY{n}{UpSampling2D}\PY{p}{(}\PY{p}{(}\PY{l+m+mi}{2}\PY{p}{,} \PY{l+m+mi}{2}\PY{p}{)}\PY{p}{)}\PY{p}{,}
    \PY{n}{layers}\PY{o}{.}\PY{n}{Conv2D}\PY{p}{(}\PY{l+m+mi}{32}\PY{p}{,} \PY{p}{(}\PY{l+m+mi}{3}\PY{p}{,} \PY{l+m+mi}{3}\PY{p}{)}\PY{p}{,} \PY{n}{activation}\PY{o}{=}\PY{l+s+s1}{\PYZsq{}}\PY{l+s+s1}{relu}\PY{l+s+s1}{\PYZsq{}}\PY{p}{,} \PY{n}{padding}\PY{o}{=}\PY{l+s+s1}{\PYZsq{}}\PY{l+s+s1}{same}\PY{l+s+s1}{\PYZsq{}}\PY{p}{)}\PY{p}{,}
    \PY{n}{layers}\PY{o}{.}\PY{n}{UpSampling2D}\PY{p}{(}\PY{p}{(}\PY{l+m+mi}{2}\PY{p}{,} \PY{l+m+mi}{2}\PY{p}{)}\PY{p}{)}\PY{p}{,}
    \PY{n}{layers}\PY{o}{.}\PY{n}{Conv2D}\PY{p}{(}\PY{l+m+mi}{1}\PY{p}{,} \PY{p}{(}\PY{l+m+mi}{3}\PY{p}{,} \PY{l+m+mi}{3}\PY{p}{)}\PY{p}{,} \PY{n}{activation}\PY{o}{=}\PY{l+s+s1}{\PYZsq{}}\PY{l+s+s1}{sigmoid}\PY{l+s+s1}{\PYZsq{}}\PY{p}{,} \PY{n}{padding}\PY{o}{=}\PY{l+s+s1}{\PYZsq{}}\PY{l+s+s1}{same}\PY{l+s+s1}{\PYZsq{}}\PY{p}{)}
\PY{p}{]}\PY{p}{)}

\PY{c+c1}{\PYZsh{} 编译模型}
\PY{n}{model}\PY{o}{.}\PY{n}{compile}\PY{p}{(}\PY{n}{optimizer}\PY{o}{=}\PY{l+s+s1}{\PYZsq{}}\PY{l+s+s1}{adam}\PY{l+s+s1}{\PYZsq{}}\PY{p}{,}
              \PY{n}{loss}\PY{o}{=}\PY{l+s+s1}{\PYZsq{}}\PY{l+s+s1}{binary\PYZus{}crossentropy}\PY{l+s+s1}{\PYZsq{}}\PY{p}{,}
              \PY{n}{metrics}\PY{o}{=}\PY{p}{[}\PY{l+s+s1}{\PYZsq{}}\PY{l+s+s1}{accuracy}\PY{l+s+s1}{\PYZsq{}}\PY{p}{]}\PY{p}{)}
\end{Verbatim}
\end{tcolorbox}

    \begin{tcolorbox}[breakable, size=fbox, boxrule=1pt, pad at break*=1mm,colback=cellbackground, colframe=cellborder]
\prompt{In}{incolor}{76}{\boxspacing}
\begin{Verbatim}[commandchars=\\\{\}]
\PY{c+c1}{\PYZsh{} 可视化示例数据}
\PY{n}{plt}\PY{o}{.}\PY{n}{figure}\PY{p}{(}\PY{n}{figsize}\PY{o}{=}\PY{p}{(}\PY{l+m+mi}{12}\PY{p}{,} \PY{l+m+mi}{4}\PY{p}{)}\PY{p}{)}
\PY{k}{for} \PY{n}{i} \PY{o+ow}{in} \PY{n+nb}{range}\PY{p}{(}\PY{l+m+mi}{4}\PY{p}{)}\PY{p}{:}
    \PY{n}{plt}\PY{o}{.}\PY{n}{subplot}\PY{p}{(}\PY{l+m+mi}{1}\PY{p}{,} \PY{l+m+mi}{4}\PY{p}{,} \PY{n}{i} \PY{o}{+} \PY{l+m+mi}{1}\PY{p}{)}
    \PY{k}{if} \PY{n}{i} \PY{o}{==} \PY{l+m+mi}{0}\PY{p}{:}
        \PY{n}{plt}\PY{o}{.}\PY{n}{title}\PY{p}{(}\PY{l+s+s1}{\PYZsq{}}\PY{l+s+s1}{Input Image}\PY{l+s+s1}{\PYZsq{}}\PY{p}{)}
        \PY{n}{plt}\PY{o}{.}\PY{n}{imshow}\PY{p}{(}\PY{n}{input\PYZus{}data}\PY{p}{[}\PY{l+m+mi}{0}\PY{p}{]}\PY{p}{)}
    \PY{k}{else}\PY{p}{:}
        \PY{n}{plt}\PY{o}{.}\PY{n}{title}\PY{p}{(}\PY{l+s+sa}{f}\PY{l+s+s1}{\PYZsq{}}\PY{l+s+s1}{Label }\PY{l+s+si}{\PYZob{}}\PY{n}{i}\PY{l+s+si}{\PYZcb{}}\PY{l+s+s1}{\PYZsq{}}\PY{p}{)}
        \PY{n}{plt}\PY{o}{.}\PY{n}{imshow}\PY{p}{(}\PY{n}{labels}\PY{p}{[}\PY{l+m+mi}{0}\PY{p}{,} \PY{p}{:}\PY{p}{,} \PY{p}{:}\PY{p}{]}\PY{p}{,} \PY{n}{cmap}\PY{o}{=}\PY{l+s+s1}{\PYZsq{}}\PY{l+s+s1}{gray}\PY{l+s+s1}{\PYZsq{}}\PY{p}{)}
    \PY{n}{plt}\PY{o}{.}\PY{n}{axis}\PY{p}{(}\PY{l+s+s1}{\PYZsq{}}\PY{l+s+s1}{off}\PY{l+s+s1}{\PYZsq{}}\PY{p}{)}

\PY{n}{plt}\PY{o}{.}\PY{n}{show}\PY{p}{(}\PY{p}{)}

\PY{c+c1}{\PYZsh{} 训练模型}
\PY{n}{history} \PY{o}{=} \PY{n}{model}\PY{o}{.}\PY{n}{fit}\PY{p}{(}\PY{n}{input\PYZus{}data}\PY{p}{,} \PY{n}{labels}\PY{p}{,} \PY{n}{epochs}\PY{o}{=}\PY{l+m+mi}{10}\PY{p}{,} \PY{n}{validation\PYZus{}split}\PY{o}{=}\PY{l+m+mf}{0.2}\PY{p}{)}

\PY{c+c1}{\PYZsh{} 可视化训练和验证损失}
\PY{n}{plt}\PY{o}{.}\PY{n}{figure}\PY{p}{(}\PY{n}{figsize}\PY{o}{=}\PY{p}{(}\PY{l+m+mi}{12}\PY{p}{,} \PY{l+m+mi}{4}\PY{p}{)}\PY{p}{)}
\PY{n}{plt}\PY{o}{.}\PY{n}{subplot}\PY{p}{(}\PY{l+m+mi}{1}\PY{p}{,} \PY{l+m+mi}{2}\PY{p}{,} \PY{l+m+mi}{1}\PY{p}{)}
\PY{n}{plt}\PY{o}{.}\PY{n}{plot}\PY{p}{(}\PY{n}{history}\PY{o}{.}\PY{n}{history}\PY{p}{[}\PY{l+s+s1}{\PYZsq{}}\PY{l+s+s1}{loss}\PY{l+s+s1}{\PYZsq{}}\PY{p}{]}\PY{p}{,} \PY{n}{label}\PY{o}{=}\PY{l+s+s1}{\PYZsq{}}\PY{l+s+s1}{Training Loss}\PY{l+s+s1}{\PYZsq{}}\PY{p}{)}
\PY{n}{plt}\PY{o}{.}\PY{n}{plot}\PY{p}{(}\PY{n}{history}\PY{o}{.}\PY{n}{history}\PY{p}{[}\PY{l+s+s1}{\PYZsq{}}\PY{l+s+s1}{val\PYZus{}loss}\PY{l+s+s1}{\PYZsq{}}\PY{p}{]}\PY{p}{,} \PY{n}{label}\PY{o}{=}\PY{l+s+s1}{\PYZsq{}}\PY{l+s+s1}{Validation Loss}\PY{l+s+s1}{\PYZsq{}}\PY{p}{)}
\PY{n}{plt}\PY{o}{.}\PY{n}{xlabel}\PY{p}{(}\PY{l+s+s1}{\PYZsq{}}\PY{l+s+s1}{Epochs}\PY{l+s+s1}{\PYZsq{}}\PY{p}{)}
\PY{n}{plt}\PY{o}{.}\PY{n}{ylabel}\PY{p}{(}\PY{l+s+s1}{\PYZsq{}}\PY{l+s+s1}{Loss}\PY{l+s+s1}{\PYZsq{}}\PY{p}{)}
\PY{n}{plt}\PY{o}{.}\PY{n}{legend}\PY{p}{(}\PY{p}{)}

\PY{c+c1}{\PYZsh{} 可视化训练和验证准确率}
\PY{n}{plt}\PY{o}{.}\PY{n}{subplot}\PY{p}{(}\PY{l+m+mi}{1}\PY{p}{,} \PY{l+m+mi}{2}\PY{p}{,} \PY{l+m+mi}{2}\PY{p}{)}
\PY{n}{plt}\PY{o}{.}\PY{n}{plot}\PY{p}{(}\PY{n}{history}\PY{o}{.}\PY{n}{history}\PY{p}{[}\PY{l+s+s1}{\PYZsq{}}\PY{l+s+s1}{accuracy}\PY{l+s+s1}{\PYZsq{}}\PY{p}{]}\PY{p}{,} \PY{n}{label}\PY{o}{=}\PY{l+s+s1}{\PYZsq{}}\PY{l+s+s1}{Training Accuracy}\PY{l+s+s1}{\PYZsq{}}\PY{p}{)}
\PY{n}{plt}\PY{o}{.}\PY{n}{plot}\PY{p}{(}\PY{n}{history}\PY{o}{.}\PY{n}{history}\PY{p}{[}\PY{l+s+s1}{\PYZsq{}}\PY{l+s+s1}{val\PYZus{}accuracy}\PY{l+s+s1}{\PYZsq{}}\PY{p}{]}\PY{p}{,} \PY{n}{label}\PY{o}{=}\PY{l+s+s1}{\PYZsq{}}\PY{l+s+s1}{Validation Accuracy}\PY{l+s+s1}{\PYZsq{}}\PY{p}{)}
\PY{n}{plt}\PY{o}{.}\PY{n}{xlabel}\PY{p}{(}\PY{l+s+s1}{\PYZsq{}}\PY{l+s+s1}{Epochs}\PY{l+s+s1}{\PYZsq{}}\PY{p}{)}
\PY{n}{plt}\PY{o}{.}\PY{n}{ylabel}\PY{p}{(}\PY{l+s+s1}{\PYZsq{}}\PY{l+s+s1}{Accuracy}\PY{l+s+s1}{\PYZsq{}}\PY{p}{)}
\PY{n}{plt}\PY{o}{.}\PY{n}{legend}\PY{p}{(}\PY{p}{)}

\PY{n}{plt}\PY{o}{.}\PY{n}{show}\PY{p}{(}\PY{p}{)}
\end{Verbatim}
\end{tcolorbox}

    \begin{center}
    \adjustimage{max size={0.9\linewidth}{0.9\paperheight}}{output_23_0.png}
    \end{center}
    { \hspace*{\fill} \\}
    
    \begin{Verbatim}[commandchars=\\\{\}]
Epoch 1/10
5/5 [==============================] - 16s 3s/step - loss: 0.5181 - accuracy:
0.7332 - val\_loss: 0.6254 - val\_accuracy: 0.4883
Epoch 2/10
5/5 [==============================] - 16s 3s/step - loss: 0.5449 - accuracy:
0.7072 - val\_loss: 0.4760 - val\_accuracy: 0.6200
Epoch 3/10
5/5 [==============================] - 17s 3s/step - loss: 0.5391 - accuracy:
0.7085 - val\_loss: 0.5894 - val\_accuracy: 0.5041
Epoch 4/10
5/5 [==============================] - 16s 3s/step - loss: 0.5203 - accuracy:
0.7333 - val\_loss: 0.5155 - val\_accuracy: 0.5622
Epoch 5/10
5/5 [==============================] - 17s 3s/step - loss: 0.5187 - accuracy:
0.7375 - val\_loss: 0.5254 - val\_accuracy: 0.5532
Epoch 6/10
5/5 [==============================] - 18s 4s/step - loss: 0.5144 - accuracy:
0.7455 - val\_loss: 0.5349 - val\_accuracy: 0.5685
Epoch 7/10
5/5 [==============================] - 17s 3s/step - loss: 0.5143 - accuracy:
0.7389 - val\_loss: 0.4910 - val\_accuracy: 0.6292
Epoch 8/10
5/5 [==============================] - 17s 3s/step - loss: 0.5125 - accuracy:
0.7272 - val\_loss: 0.6707 - val\_accuracy: 0.4732
Epoch 9/10
5/5 [==============================] - 18s 3s/step - loss: 0.5099 - accuracy:
0.7436 - val\_loss: 0.4323 - val\_accuracy: 0.7050
Epoch 10/10
5/5 [==============================] - 17s 3s/step - loss: 0.5068 - accuracy:
0.7509 - val\_loss: 0.5502 - val\_accuracy: 0.5864
    \end{Verbatim}

    \begin{center}
    \adjustimage{max size={0.9\linewidth}{0.9\paperheight}}{output_23_2.png}
    \end{center}
    { \hspace*{\fill} \\}
    
    \begin{tcolorbox}[breakable, size=fbox, boxrule=1pt, pad at break*=1mm,colback=cellbackground, colframe=cellborder]
\prompt{In}{incolor}{77}{\boxspacing}
\begin{Verbatim}[commandchars=\\\{\}]
\PY{c+c1}{\PYZsh{} 将测试集影像转为数组}
\PY{k+kn}{import} \PY{n+nn}{os}
\PY{k+kn}{import} \PY{n+nn}{cv2}
\PY{k+kn}{import} \PY{n+nn}{numpy} \PY{k}{as} \PY{n+nn}{np}

\PY{c+c1}{\PYZsh{} 指定包含图像的文件夹路径}
\PY{n}{folder\PYZus{}path} \PY{o}{=} \PY{l+s+s1}{\PYZsq{}}\PY{l+s+s1}{C:}\PY{l+s+s1}{\PYZbs{}}\PY{l+s+s1}{PHD}\PY{l+s+s1}{\PYZbs{}}\PY{l+s+s1}{deeplearning\PYZus{}3}\PY{l+s+s1}{\PYZbs{}}\PY{l+s+s1}{model\PYZus{}data}\PY{l+s+s1}{\PYZbs{}}\PY{l+s+s1}{image\PYZus{}test}\PY{l+s+s1}{\PYZsq{}}

\PY{c+c1}{\PYZsh{} 初始化一个空列表来存储图像数组}
\PY{n}{image\PYZus{}data} \PY{o}{=} \PY{p}{[}\PY{p}{]}

\PY{c+c1}{\PYZsh{} 遍历文件夹中的所有图像文件}
\PY{k}{for} \PY{n}{filename} \PY{o+ow}{in} \PY{n}{os}\PY{o}{.}\PY{n}{listdir}\PY{p}{(}\PY{n}{folder\PYZus{}path}\PY{p}{)}\PY{p}{:}
    \PY{k}{if} \PY{n}{filename}\PY{o}{.}\PY{n}{endswith}\PY{p}{(}\PY{p}{(}\PY{l+s+s1}{\PYZsq{}}\PY{l+s+s1}{.jpg}\PY{l+s+s1}{\PYZsq{}}\PY{p}{,} \PY{l+s+s1}{\PYZsq{}}\PY{l+s+s1}{.jpeg}\PY{l+s+s1}{\PYZsq{}}\PY{p}{,} \PY{l+s+s1}{\PYZsq{}}\PY{l+s+s1}{.png}\PY{l+s+s1}{\PYZsq{}}\PY{p}{,}\PY{l+s+s2}{\PYZdq{}}\PY{l+s+s2}{.tif}\PY{l+s+s2}{\PYZdq{}}\PY{p}{)}\PY{p}{)}\PY{p}{:}  \PY{c+c1}{\PYZsh{} 仅处理特定图像文件类型}
        \PY{n}{file\PYZus{}path} \PY{o}{=} \PY{n}{os}\PY{o}{.}\PY{n}{path}\PY{o}{.}\PY{n}{join}\PY{p}{(}\PY{n}{folder\PYZus{}path}\PY{p}{,} \PY{n}{filename}\PY{p}{)}
        \PY{n}{image} \PY{o}{=} \PY{n}{cv2}\PY{o}{.}\PY{n}{imread}\PY{p}{(}\PY{n}{file\PYZus{}path}\PY{p}{)}  \PY{c+c1}{\PYZsh{} 使用OpenCV读取图像}
        \PY{k}{if} \PY{n}{image} \PY{o+ow}{is} \PY{o+ow}{not} \PY{k+kc}{None}\PY{p}{:}
            \PY{n}{image\PYZus{}data}\PY{o}{.}\PY{n}{append}\PY{p}{(}\PY{n}{image}\PY{p}{)}

\PY{c+c1}{\PYZsh{} 将图像数组转换为NumPy数组}
\PY{n}{image\PYZus{}data} \PY{o}{=} \PY{n}{np}\PY{o}{.}\PY{n}{array}\PY{p}{(}\PY{n}{image\PYZus{}data}\PY{p}{)}

\PY{c+c1}{\PYZsh{} 打印数组的形状,这将显示图像数量、高度、宽度和通道数}
\PY{n+nb}{print}\PY{p}{(}\PY{l+s+s2}{\PYZdq{}}\PY{l+s+s2}{Shape of image data array:}\PY{l+s+s2}{\PYZdq{}}\PY{p}{,} \PY{n}{image\PYZus{}data}\PY{o}{.}\PY{n}{shape}\PY{p}{)}
\end{Verbatim}
\end{tcolorbox}

    \begin{Verbatim}[commandchars=\\\{\}]
Shape of image data array: (41, 256, 256, 3)
    \end{Verbatim}

    \begin{tcolorbox}[breakable, size=fbox, boxrule=1pt, pad at break*=1mm,colback=cellbackground, colframe=cellborder]
\prompt{In}{incolor}{78}{\boxspacing}
\begin{Verbatim}[commandchars=\\\{\}]
\PY{n}{input\PYZus{}data\PYZus{}test} \PY{o}{=} \PY{n}{image\PYZus{}data} \PY{o}{/} \PY{l+m+mf}{255.0}
\end{Verbatim}
\end{tcolorbox}

    \begin{tcolorbox}[breakable, size=fbox, boxrule=1pt, pad at break*=1mm,colback=cellbackground, colframe=cellborder]
\prompt{In}{incolor}{81}{\boxspacing}
\begin{Verbatim}[commandchars=\\\{\}]
\PY{n}{predictions} \PY{o}{=} \PY{n}{model}\PY{o}{.}\PY{n}{predict}\PY{p}{(}\PY{n}{input\PYZus{}data\PYZus{}test}\PY{p}{)}
\end{Verbatim}
\end{tcolorbox}

    \begin{Verbatim}[commandchars=\\\{\}]
2/2 [==============================] - 1s 129ms/step
    \end{Verbatim}

    \begin{tcolorbox}[breakable, size=fbox, boxrule=1pt, pad at break*=1mm,colback=cellbackground, colframe=cellborder]
\prompt{In}{incolor}{83}{\boxspacing}
\begin{Verbatim}[commandchars=\\\{\}]
\PY{n}{predictions}\PY{o}{.}\PY{n}{shape}
\end{Verbatim}
\end{tcolorbox}

            \begin{tcolorbox}[breakable, size=fbox, boxrule=.5pt, pad at break*=1mm, opacityfill=0]
\prompt{Out}{outcolor}{83}{\boxspacing}
\begin{Verbatim}[commandchars=\\\{\}]
(41, 256, 256, 1)
\end{Verbatim}
\end{tcolorbox}
        
    \begin{tcolorbox}[breakable, size=fbox, boxrule=1pt, pad at break*=1mm,colback=cellbackground, colframe=cellborder]
\prompt{In}{incolor}{84}{\boxspacing}
\begin{Verbatim}[commandchars=\\\{\}]
\PY{k+kn}{import} \PY{n+nn}{pandas} \PY{k}{as} \PY{n+nn}{pd}
\PY{n}{predicted\PYZus{}classes} \PY{o}{=} \PY{n}{tf}\PY{o}{.}\PY{n}{argmax}\PY{p}{(}\PY{n}{predictions}\PY{p}{,} \PY{n}{axis}\PY{o}{=}\PY{o}{\PYZhy{}}\PY{l+m+mi}{1}\PY{p}{)}\PY{o}{.}\PY{n}{numpy}\PY{p}{(}\PY{p}{)}
\PY{n}{predicted\PYZus{}classes}\PY{o}{.}\PY{n}{shape}
\end{Verbatim}
\end{tcolorbox}

            \begin{tcolorbox}[breakable, size=fbox, boxrule=.5pt, pad at break*=1mm, opacityfill=0]
\prompt{Out}{outcolor}{84}{\boxspacing}
\begin{Verbatim}[commandchars=\\\{\}]
(41, 256, 256)
\end{Verbatim}
\end{tcolorbox}
        
    \begin{tcolorbox}[breakable, size=fbox, boxrule=1pt, pad at break*=1mm,colback=cellbackground, colframe=cellborder]
\prompt{In}{incolor}{87}{\boxspacing}
\begin{Verbatim}[commandchars=\\\{\}]
\PY{n}{predicted\PYZus{}classes}\PY{p}{[}\PY{l+m+mi}{3}\PY{p}{]}
\end{Verbatim}
\end{tcolorbox}

            \begin{tcolorbox}[breakable, size=fbox, boxrule=.5pt, pad at break*=1mm, opacityfill=0]
\prompt{Out}{outcolor}{87}{\boxspacing}
\begin{Verbatim}[commandchars=\\\{\}]
array([[0, 0, 0, {\ldots}, 0, 0, 0],
       [0, 0, 0, {\ldots}, 0, 0, 0],
       [0, 0, 0, {\ldots}, 0, 0, 0],
       {\ldots},
       [0, 0, 0, {\ldots}, 0, 0, 0],
       [0, 0, 0, {\ldots}, 0, 0, 0],
       [0, 0, 0, {\ldots}, 0, 0, 0]], dtype=int64)
\end{Verbatim}
\end{tcolorbox}
        
    \begin{tcolorbox}[breakable, size=fbox, boxrule=1pt, pad at break*=1mm,colback=cellbackground, colframe=cellborder]
\prompt{In}{incolor}{86}{\boxspacing}
\begin{Verbatim}[commandchars=\\\{\}]
\PY{n}{num\PYZus{}samples\PYZus{}to\PYZus{}visualize} \PY{o}{=} \PY{l+m+mi}{20}  \PY{c+c1}{\PYZsh{} 可视化前20个样本}
\PY{n}{plt}\PY{o}{.}\PY{n}{figure}\PY{p}{(}\PY{n}{figsize}\PY{o}{=}\PY{p}{(}\PY{l+m+mi}{15}\PY{p}{,} \PY{l+m+mi}{6}\PY{p}{)}\PY{p}{)}
\PY{k}{for} \PY{n}{i} \PY{o+ow}{in} \PY{n+nb}{range}\PY{p}{(}\PY{n}{num\PYZus{}samples\PYZus{}to\PYZus{}visualize}\PY{p}{)}\PY{p}{:}
    \PY{n}{plt}\PY{o}{.}\PY{n}{subplot}\PY{p}{(}\PY{l+m+mi}{4}\PY{p}{,} \PY{l+m+mi}{5}\PY{p}{,} \PY{n}{i} \PY{o}{+} \PY{l+m+mi}{1}\PY{p}{)}
    \PY{n}{plt}\PY{o}{.}\PY{n}{imshow}\PY{p}{(}\PY{n}{predicted\PYZus{}classes}\PY{p}{[}\PY{n}{i}\PY{p}{]}\PY{p}{)}
    \PY{n}{plt}\PY{o}{.}\PY{n}{title}\PY{p}{(}\PY{l+s+sa}{f}\PY{l+s+s1}{\PYZsq{}}\PY{l+s+s1}{Sample }\PY{l+s+si}{\PYZob{}}\PY{n}{i}\PY{+w}{ }\PY{o}{+}\PY{+w}{ }\PY{l+m+mi}{1}\PY{l+s+si}{\PYZcb{}}\PY{l+s+s1}{\PYZsq{}}\PY{p}{)}
    \PY{n}{plt}\PY{o}{.}\PY{n}{axis}\PY{p}{(}\PY{l+s+s1}{\PYZsq{}}\PY{l+s+s1}{off}\PY{l+s+s1}{\PYZsq{}}\PY{p}{)}

\PY{n}{plt}\PY{o}{.}\PY{n}{show}\PY{p}{(}\PY{p}{)}
\end{Verbatim}
\end{tcolorbox}

    \begin{center}
    \adjustimage{max size={0.9\linewidth}{0.9\paperheight}}{output_30_0.png}
    \end{center}
    { \hspace*{\fill} \\}
    
    \begin{tcolorbox}[breakable, size=fbox, boxrule=1pt, pad at break*=1mm,colback=cellbackground, colframe=cellborder]
\prompt{In}{incolor}{ }{\boxspacing}
\begin{Verbatim}[commandchars=\\\{\}]

\end{Verbatim}
\end{tcolorbox}


    % Add a bibliography block to the postdoc
    
    
    
\end{document}
